\documentclass[12pt]{article}
\usepackage{amsmath,amssymb,amsthm}
\usepackage{geometry}
\usepackage[round]{natbib}
\usepackage{graphicx}
\usepackage{float}
\usepackage{booktabs} 
\usepackage{array}
\usepackage{tcolorbox}

\geometry{margin=1in}
\newcommand{\KL}{\mathrm{KL}}
\newcommand{\tr}{\mathrm{tr}}
\newcommand{\Sig}{\Sigma}
\newcommand{\SigQ}{\Sigma^q}
\newcommand{\SigP}{\Sigma^p}
\newcommand{\muQ}{\mu^q}
\newcommand{\muP}{\mu^p}
\newcommand{\Dmu}{\Delta\mu}
\newcommand{\vech}{\mathrm{vech}}
\newcommand{\Prec}{P}
\newtheorem{theorem}{Theorem}
\newtheorem{proposition}[theorem]{Proposition}
\newtheorem{definition}[theorem]{Definition}
\newtheorem{prediction}[theorem]{Prediction}

\title{The Inertia of Belief}

\author{
Robert C. Dennis\\
Independent Researcher\\
Leander, Texas 78641\\
\texttt{cdenn016@gmail.com}
}

\date{\today}

\begin{document}
\maketitle


\begin{abstract}
We present a dynamical theory of belief evolution where cognitive agents possess epistemic momentum and inertia proportional to an agent's prior precision and interactions with other agents are encoded via a gauge-invariant attention mechanism which allows agents to communicate and interpret beliefs via internal perspectival frames of reference. We show that the second-order Taylor expansion of the KL divergence produces the Fisher information metric as an inertial mass tensor for belief dynamics. This extends the variational free energy principle from first-order gradient descent to a second-order dynamics formally analogous to Hamiltonian mechanics where confident beliefs resist change while uncertain beliefs update readily. Standard Bayesian updating emerges as the over-damped limit of our theory. In under-damped regimes, the framework predicts phenomena absent from first-order models such as  belief oscillation, over-shooting, and epistemic resonance. We derive specific, falsifiable predictions distinguishing this framework from conventional approaches: (1) confident beliefs should overshoot equilibria and oscillate when confronted with strong opposing evidence, (2) belief relaxation times scale with prior precision, and (3) periodic persuasion achieves maximum effect at a characteristic epistemic resonance frequency $\omega = \sqrt{\text{precision} \times \text{evidence strength}}$. We validate these predictions through numerical simulation and propose experiments for empirical testing. Our framework offers a geometric explanation for confirmation bias, belief perseverance, and opinion polarization as natural consequences of epistemic inertia rather than cognitive irrationality.
\end{abstract}

\noindent\textbf{Keywords:} Gauge theory $\cdot$ Active inference $\cdot$ Free energy principle $\cdot$ Information geometry $\cdot$ Sociology

\section{Introduction}

Why do some beliefs resist change more than others? Some are stiff and yet others readily sway to and fro. While confident beliefs clearly possess more "inertia" than uncertain ones, a principled mathematical foundation for this intuitive phenomenon remains elusive. Current theories of belief updating, from Bayesian inference \citep{jaynes2003probability} to predictive coding \citep{friston2010free,clark2013whatever}, model belief change as gradient descent. This is a purely dissipative process where beliefs flow toward lower free energy without momentum, inertia, or dynamics. Though enormously successful across neuroscience \citep{friston2016active}, psychology \citep{hohwy2013predictive}, and machine learning \citep{millidge2021predictive}, this framework fundamentally remains incomplete.

In this article, we show that beliefs possess an "epistemic" inertia proportional to an agent's prior precision. Just as physical objects with mass resist acceleration, beliefs held with high confidence (precision) resist change and, once moving, tend to continue in their direction. This is not merely metaphor but, rather, it is a mathematical consequence of a second order expansion of the variational free energy. In this view the Fisher information metric \citep{amari2016information}, which measures statistical distinguishability, simultaneously provides an inertial mass tensor for belief dynamics. The second-order terms in the KL divergence expansion, traditionally neglected due to myriad reasons\citep{friston2008hierarchical,bogacz2017tutorial}, generate rich Hamiltonian dynamics with conserved quantities.

Furthermore, beliefs propagate through networks of agents in attention patterns ranging from coordinated consensus to turbulent disagreement, often exhibiting distortion, resonance, and phase transitions \citep{castellano2009statistical,galam2012sociophysics}. While numerous models, from opinion dynamics \citep{hegselmann2002opinion} to quantum-inspired approaches \citep{busemeyer2012quantum}, capture aspects of collective belief evolution, a principled geometric foundation remains incomplete and wholly absent.

As an intuitive example, consider an agent with strong priors about a political position (high precision). When presented with contradicting evidence, their belief doesn't immediately flip but instead resists change (inertia), may overshoot when it does shift (momentum), and might oscillate before settling (under-damped dynamics). Conversely, an uncertain agent (low prior precision) updates quickly toward new evidence or observation with minimal resistance. These phenomena, typically attributed to cognitive biases \citep{kahneman2011thinking}, emerge naturally from belief inertia.

Our framework makes three contributions to the field:

\begin{enumerate}
\item \textbf{Theoretical}: We derive a second order belief dynamics from first principles, showing the Fisher metric provides a natural inertial mass tensor $M = \Sigma_p^{-1} = \Lambda_p$ (prior precision). Via pullback geometry on informational bundles \citep{amari2016information,nielsen2020elementary}, we extend variational free energy to multi-agent systems characterized by belief-momentum exchange and gauge-covariant transport between agents in attention patterns where agents can hold identical beliefs yet have distinct perspectives\cite{vaswani2017attention} \cite{Dennis2025}\citep{kobayashi1963foundations}.

\item \textbf{Phenomenological}: We predict novel cognitive and social phenomena including belief oscillations, overshooting, and resonance emerging from belief inertia. These effects are absent in first order treatments \citep{parr2022active} yet provide testable predictions distinguishing our framework from purely dissipative models for a variety of informational systems.

\item \textbf{Psychological}: We argue that cognitive biases such as confirmation bias \citep{nickerson1998confirmation}, Dunning-Kruger effect \citep{kruger1999unskilled}, and belief perseverance \citep{anderson1980perseverance}, and more are natural consequences of belief inertia rather than irrationality, offering a unified geometric explanation for seemingly disparate phenomena. 
\end{enumerate}

Our approach unlocks mathematical tools traditionally relegated to physics such as symplectic geometry \citep{arnold1989mathematical}, perturbation theory \citep{holmes2012introduction}, Noether's theorem \citep{olver1993applications}, renormalization group methods \citep{wilson1975renormalization,goldenfeld1992lectures}, topological phenomena \citep{nakahara2003geometry,bernevig2013topological}, and critical point analyses \citep{strogatz2015nonlinear,sornette2006critical} for understanding cognitive, social, and economic dynamics. By recognizing beliefs as dynamical objects with genuine inertia, we bridge information geometry, cognitive science, and collective behavior within a unified Hamiltonian framework.

\section{Mathematical Framework}

\subsection{Beliefs as Points on Statistical Manifolds}
We model beliefs as probability distributions $q(\theta)$ parameterized by $\theta \in \mathbb{R}^n$ on a statistical manifold $\mathcal{M}$. 

For the remainer of this article we shall consider multio-variate Gaussian (MVG) beliefs and priors

\begin{equation}
q = \mathcal{N}(\mu_q, \Sigma_q)
p = \mathcal{N}(\mu_p, \Sigma_p)
\end{equation}

where $\mu_\nu$ represents the believed value and $\Sigma_\nu$ represents uncertainty.

The Kullback-Leibler (KL) divergence measures the epistemic distance between an agent's belief $q$ and their prior model $p$

\begin{equation}
\text{KL}(q \| p) = \int q(x) \log \frac{q(x)}{p(x)} dx
\end{equation}



\subsection{Multi-Agent Belief Geometry}

We extend our single-agent framework to networks of interacting cognitive agents via attention. Following \cite{Dennis2025}, we model agents as residing on a gauge-theoretic bundle geometry where each agent $i$ maintains beliefs and priors $q_i = \mathcal{N}(\mu_i, \Sigma_i)$ as well as an internal reference frame $\phi_i$ that determine how they interpret information.

Importantly, agents cannot directly compare beliefs. Instead, they must first align their gauge frames via parallel transport operators given by

\begin{equation}
\Omega_{ij} = e^{\phi_i}e^{-\phi_j}
\end{equation}

 

This operator transforms agent $j$'s beliefs into agent $i$'s gauge frame of reference. This gauge structure formalizes the fundamental psychological reality that agents cannot directly share beliefs but must translate them through their respective internal interpretive perspectives. Importantly, flat gauge reproduces standard consensus models.

This operator acts by right action as

\begin{equation}
q_j \to \Omega_{ij} \cdot q_j = \mathcal{N}(\Omega_{ij}\mu_j, \Omega_{ij}\Sigma_j\Omega_{ij}^T)
\end{equation} 

For example, it may be helpful to consider $\Omega_{ij} \in SO(3)$, the group of rotations where $\phi \in \mathfrak{so}3$, the Lie algebra of SO(3).

The transformed belief can then be compared with agent $i$'s own beliefs via KL divergence

\begin{equation}
D_{ij} = D_{\mathrm{KL}}(q_i \| \Omega_{ij} \cdot q_j)
\end{equation}

Notice that this transport is, in general, asymmetric.

\subsection{Multi-Agent Free Energy}

As we derive in full detail in \cite{Dennis2025} the total variational free energy for a network of agents balances individual belief maintenance with social consensus pressure as

\begin{align}
\mathcal{F}[\{q_i\}, \{\phi_i\}] &= \sum_i \underbrace{D_{\mathrm{KL}}(q_i \| p_i)}_{\text{Prior beliefs}} + \sum_{i,j} \underbrace{\beta_{ij} D_{\mathrm{KL}}(q_i \| \Omega_{ij} \cdot q_j)}_{\text{Social alignment}} \\
&\quad - \sum_i \underbrace{\mathbb{E}_{q_i}[\log p(o_i \mid \mu_i)]}_{\text{Sensory evidence}}
\end{align}

where $\beta_{ij}$ represents the attention agent $i$ places in agent $j$'s beliefs and we take $p_i$ to be quasi-static. The attention naturally emerges as

\begin{equation}
\beta_{ij} = \frac{\exp(-D_{\mathrm{KL}}(q_i \| \Omega_{ij} \cdot q_j)/\tau)}{\sum_k \exp(-D_{\mathrm{KL}}(q_i \| \Omega_{ik} \cdot q_k)/\tau)}
\end{equation}

with temperature $\tau$ controlling selectivity (recovering transformer attention mechanisms \cite{Dennis2025}). In previous work we have shown that sensory evidence and/or observations are equivalent to agent-agent attention coupling.  Hence, we will no longer utilize $\mathbb{E}_{q_i}[\log p(o_i \mid \mu_i)]$ and instead consider only the first two terms of the variational multi-agent free energy \cite{Dennis2025a}



%==============================================================================
\subsection{Hamiltonian Formulation of Belief Dynamics}
\label{sec:hamiltonian}
%==============================================================================

The variational free energy principle is typically formulated as gradient descent—a purely dissipative dynamics where beliefs flow downhill toward equilibrium. However, this picture is incomplete. The second-order Taylor expansion of KL divergence reveals a kinetic energy term systematically neglected in standard treatments, extending the free energy principle from gradient flow to fully conservative Hamiltonian mechanics. This extension has profound implications: beliefs acquire inertia, cognitive systems exhibit momentum, and the Fisher information metric emerges as a mass matrix identifying \emph{precision with inertial mass}.

%------------------------------------------------------------------------------
\subsection{The Adiabatic Approximation}
%------------------------------------------------------------------------------

Cognitive systems operate across multiple timescales often hierarchically. Beliefs generally update rapidly in response to sensory evidence when compared to priors which encode stable world-views, personality traits, or cultural assumptions.  These evolve slowly through learning and interaction. We formalize this separation via the adiabatic approximation

Let the prior parameters $(\bar{\mu}_i, \bar{\Sigma}_i)$ evolve on a slow timescale $T$, while beliefs $(\mu_i, \Sigma_i)$ evolve on a fast timescale $t$, with $\epsilon = t/T \ll 1$. 


In the quasi-static limit $\epsilon \to 0$

\begin{itemize}
    \item Priors $(\bar{\mu}_i, \bar{\Sigma}_i)$ are treated as fixed external parameters
    \item Only beliefs $(\mu_i, \Sigma_i)$ are dynamical variables
    \item The configuration space reduces to $\mathcal{Q} = \prod_i [\mathbb{R}^d \times \mathrm{SPD}(d)]$
\end{itemize}

This approximation captures the phenomenology of rapid belief inference against a stable anchor of learned expectations and behaviors. The slow drift of priors toward equilibrated beliefs constitutes learning.

%------------------------------------------------------------------------------
\subsection{State Space and Phase Space}
%------------------------------------------------------------------------------

Each agent $i$ maintains a Gaussian belief $q_i = \mathcal{N}(\mu_i, \Sigma_i)$ anchored to a fixed prior $p_i = \mathcal{N}(\bar{\mu}_i, \bar{\Sigma}_i)$. The dynamical state vector is then

\begin{equation}
\xi_i = (\mu_i, \Sigma_i) \in \mathbb{R}^d \times \mathrm{SPD}(d)
\end{equation}

with dimension $d + \frac{d(d+1)}{2} = \frac{d(d+3)}{2}$ per agent.

The full system for $N$ agents is $\xi = (\xi_1, \ldots, \xi_N)$, living on the product manifold

\begin{equation}
\mathcal{Q} = \prod_{i=1}^N \left[\mathbb{R}^d \times \mathrm{SPD}(d)\right]
\end{equation}

To formulate Hamiltonian mechanics, we introduce \textbf{conjugate momenta}

\begin{align}
\pi_i^\mu &\in \mathbb{R}^d & &\text{(momentum conjugate to mean)} \\
\Pi_i^\Sigma &\in \mathrm{Sym}(d) & &\text{(momentum conjugate to covariance)}
\end{align}

The phase space is then the cotangent bundle $T^*\mathcal{Q}$ where $(\xi, \pi) = (\mu, \Sigma, \pi^\mu, \Pi^\Sigma)$.

%------------------------------------------------------------------------------
\subsection{The Reduced Free Energy}
%------------------------------------------------------------------------------

With priors fixed, the free energy becomes a functional of beliefs only

\begin{equation}
\boxed{
F[\{q_i\}] = \sum_i \mathrm{KL}(q_i \| p_i) + \sum_{i,j} \beta_{ij} \, \mathrm{KL}(q_i \| \Omega_{ij}[q_j]) 
}
\end{equation}

The first term pegs each agent's belief to its prior and the second aligns beliefs across the social network through gauge-covariant transport $\Omega_{ij}$ and attention. This unified form subsumes observation terms via the freedom to impose that sensory data simply constitute messages from "environmental agents" with no ontological distinction. \cite{Dennis2025}

%------------------------------------------------------------------------------
\subsection{Mass as Precision: The Fisher-Rao Metric}
%------------------------------------------------------------------------------

The primary result enabling Hamiltonian mechanics is that the Hessian of free energy serves as the inertia/mass matrix

\begin{equation}
\mathbf{M} = \frac{\partial^2 F}{\partial\xi\partial\xi} = \mathcal{G}
\end{equation}

where $\mathcal{G}$ is the Fisher-Rao information metric on the statistical manifold.

\subsubsection{Mean Sector Mass Matrix}

For the $\mu$ parameters, this mass matrix has block structure

\begin{equation}
[\mathbf{M}^\mu]_{ik} = 
\begin{cases}
\displaystyle \bar{\Lambda}_{pi} + \sum_l \beta_{il}\tilde{\Lambda}_{ql} + \sum_j \beta_{ji}\Lambda_{qi} & i = k \\[12pt]
\displaystyle -\beta_{ik}\Omega_{ik}\Lambda_{qk} - \beta_{ki}\Lambda_{qi}\Omega_{ki}^T & i \neq k
\end{cases}
\end{equation}

where $\Lambda_{qi} = \Sigma_{qi}^{-1}$ is belief precision, $\bar{\Lambda}_{pi} = \bar{\Sigma}_{pi}^{-1}$ is prior precision, and $\tilde{\Lambda}_{qk} = \Omega_{ik}\Lambda_{qk}\Omega_{ik}^T$ is transported precision.

The diagonal block defines the effective mass of agent $i$

\begin{equation}
\boxed{
M_i = \bar{\Lambda}_i + \sum_k \beta_{ik}\tilde{\Lambda}_k + \sum_j \beta_{ji}\Lambda_i
}
\end{equation}

This three-part structure has clear physical interpretation:
\begin{itemize}
    \item $\bar{\Lambda}_i$: Bare mass from prior precision—inertia against deviation from deep expectations
    \item $\sum_k \beta_{ik}\tilde{\Lambda}_k$: Incoming relational mass—inertia from being ``pulled'' by neighbors
    \item $\sum_j \beta_{ji}\Lambda_i$: Outgoing relational mass—recoil from "pulling" neighbors
\end{itemize}

The off-diagonal blocks $[\mathbf{M}^\mu]_{ik}$ encode a kinetic coupling.  This means that when agent $k$ accelerates, agent $i$ experiences a correlated force proportional to coupling strength and relative precision.

\subsubsection{Covariance Sector Mass Matrix}

The covariance parameters $\Sigma_i \in \mathrm{SPD}(d)$ live on a curved manifold with its own metric structure. The mass matrix in this sector is

\begin{equation}
[\mathbf{M}^\Sigma]_{ii} = \frac{1}{2}(\Lambda_i \otimes \Lambda_i) \cdot \mathcal{S} \cdot \left(1 + \sum_k \beta_{ik} + \sum_j \beta_{ji}\right)
\end{equation}

where $\mathcal{S}$ is the symmetrizer and $\otimes$ denotes the Kronecker product.

This gives the shape kinetic energy. This is the cost of changing one's uncertainty structure. Crucially, precision again plays the role of mass as agents with tight beliefs (high $\Lambda_i$) have large covariance inertia, resisting changes to their confidence levels.

\subsubsection{Cross Terms}

The mean/covariance cross terms $\mathbf{C}^{\mu\Sigma}$ couple position and shape dynamics as

\begin{equation}
[\mathbf{C}^{\mu\Sigma}]_{ik} = -\beta_{ik} \, \Omega_{ik}\Lambda_{qk}(\cdot)\Lambda_{qk}\Omega_{ik}^T \, (\mu_i - \tilde{\mu}_k)
\end{equation}

Importantly, these vanish when group consensus is achieved.  This occurs when $\mu_i = \tilde{\mu}_k$, i.e. the situation where agents share beliefs modulo gauge transformation. Near consensus, mean and covariance dynamics approximately decouple.

%------------------------------------------------------------------------------
\subsection{The Hamiltonian}
%------------------------------------------------------------------------------

The Hamiltonian is the total energy in the standard "kinetic + potential" form

\begin{equation}
\boxed{
H = \underbrace{\frac{1}{2}\langle\pi, \mathbf{M}^{-1}\pi\rangle}_{\text{kinetic energy}} + \underbrace{F[\xi]}_{\text{potential energy}}
}
\end{equation}

Expanding in sectors:
\begin{equation}
H = \frac{1}{2}(\pi^\mu)^T (\mathbf{M}^\mu)^{-1} \pi^\mu + \frac{1}{2}\mathrm{tr}\left[(\mathbf{M}^\Sigma)^{-1}[\Pi^\Sigma, \Pi^\Sigma]\right] + F[\mu, \Sigma]
\end{equation}

The kinetic energy has the form $T = \frac{1}{2}mv^2$ with precision playing the role of mass and belief velocity $\dot{\xi}$ related to momentum through the metric.

%------------------------------------------------------------------------------
\subsection{Hamilton's Equations: Mean Sector}
%------------------------------------------------------------------------------

The equations of motion for mean parameters follow from the canonical structure in physics

\subsubsection{Velocity Equation}
\begin{equation}
\boxed{
\dot{\mu}_i = \sum_k [\mathbf{M}^{-1}]_{ik}^{\mu\mu} \, \pi_k^\mu + \sum_k [\mathbf{M}^{-1}]_{ik}^{\mu\Sigma} \, \Pi_k^\Sigma
}
\end{equation}

This relates belief velocity to momentum through the inverse mass matrix. The off-diagonal terms show that agent $i$'s motion depends on the momenta of all coupled agents. That is to say that beliefs are not independent but kinetically entangled.

\subsubsection{Force Equation}
\begin{equation}
\boxed{
\dot{\pi}_i^\mu = -\frac{\partial F}{\partial\mu_i} - \frac{1}{2}\pi^T \frac{\partial \mathbf{M}^{-1}}{\partial\mu_i} \pi
}
\end{equation}

The first term is the potential force and represents standard first order gradient descent but here modified by our multi-agent gauge attention as

\begin{equation}
-\frac{\partial F}{\partial\mu_i} = -\bar{\Lambda}_i(\mu_i - \bar{\mu}_i) - \sum_k \beta_{ik}\tilde{\Lambda}_k(\mu_i - \tilde{\mu}_k) - \sum_j \beta_{ji}\Lambda_i\Omega_{ji}^T(\tilde{\mu}_i^{(j)} - \mu_j)
\end{equation}

This decomposes into

\begin{itemize}
    \item \textbf{Prior restoring force}: Pull toward the prior mean $\bar{\mu}_i$
    \item \textbf{Consensus force}: Pull toward transported neighbor beliefs $\tilde{\mu}_k$
    \item \textbf{Reciprocal force}: Reaction from neighbors being pulled toward agent $i$
\end{itemize}

The second term is a geodesic force arising from the state-dependent metric

\begin{equation}
f_i^{\text{geo}} = -\frac{1}{2}\sum_{jkl} (\pi_j^\mu)^T \frac{\partial [\mathbf{M}^{-1}]_{jk}^{\mu\mu}}{\partial\mu_i} \pi_k^\mu
\end{equation}

This encodes the curvature of the statistical manifold: even without potential gradients, beliefs follow curved trajectories known as the geodesics of information geometry \cite{amari2016information}.

%------------------------------------------------------------------------------
\subsection{Hamilton's Equations: Covariance Sector}
%------------------------------------------------------------------------------

The covariance sector requires extremely careful treatment since $\Sigma_i \in \mathrm{SPD}(d)$ is in a curved, hyperbolic manifold.

\subsubsection{Velocity Equation}

\begin{equation}
\boxed{
\dot{\Sigma}_i = \sum_k [\mathbf{M}^{-1}]_{ik}^{\Sigma\mu} \, \pi_k^\mu + \sum_k [\mathbf{M}^{-1}]_{ik}^{\Sigma\Sigma} \, \Pi_k^\Sigma
}
\end{equation}

The inverse metric on SPD$(d)$ takes the form

\begin{equation}
[\mathbf{M}^{-1}]^{\Sigma\Sigma}[\Pi] = 2\Sigma \, \Pi \, \Sigma \cdot (\text{normalization})
\end{equation}

This is the natural (affine-invariant) metric on the cone of positive definite matrices.

\subsubsection{Force Equation}
\begin{equation}
\boxed{
\dot{\Pi}_i^\Sigma = -\frac{\partial F}{\partial\Sigma_i} - \frac{1}{2}\pi^T \frac{\partial \mathbf{M}^{-1}}{\partial\Sigma_i} \pi
}
\end{equation}

The potential force in the covariance sector is
\begin{equation}
-\frac{\partial F}{\partial\Sigma_{qi}} = -\frac{1}{2}(\bar{\Lambda}_{pi} - \Lambda_{qi}) - \sum_k \frac{\beta_{ik}}{2}(\tilde{\Lambda}_{qk} - \Lambda_{qi})
\end{equation}

This drives covariance toward the precision-weighted average of prior and neighbor uncertainties.

%------------------------------------------------------------------------------
\subsection{Compact Form}
%------------------------------------------------------------------------------

The full system can be written compactly as

\begin{equation}
\boxed{
\begin{aligned}
\dot{\xi} &= \mathbf{M}^{-1}\pi \\[6pt]
\dot{\pi} &= -\nabla F - \frac{1}{2}\nabla_\xi\langle\pi, \mathbf{M}^{-1}\pi\rangle
\end{aligned}
}
\end{equation}

These are the informational Hamilton's equations on the cotangent bundle $T^*\mathcal{Q}$ with symplectic structure. The Hamiltonian $H$ is conserved and gauge covariant along trajectories

\begin{equation}
\frac{dH}{dt} = 0
\end{equation}

This conservation law is typically absent in gradient descent formulations but has profound implications for cognitive dynamics, allowing oscillatory and quasi-periodic behavior rather than pure relaxation.


%------------------------------------------------------------------------------
\subsection{Physical Interpretation}
%------------------------------------------------------------------------------

The Hamiltonian formulation reveals that epistemic dynamics are not merely optimization but a genuine informational mechanics on belief space

\begin{table}[ht]
\centering
\caption{Correspondence between Hamiltonian mechanics and cognitive dynamics. The free energy principle, when extended to second order, reveals that belief updating follows the same mathematical structure as classical mechanics, with precision playing the role of inertial mass.}
\label{tab:mechanics-cognition}
\renewcommand{\arraystretch}{1.4}
\begin{tabular}{@{} l l l @{}}
\toprule
\textbf{Physical Quantity} & \textbf{Cognitive Interpretation} & \textbf{Mathematical Form} \\
\midrule
Position $\mu_i$ & What agent $i$ believes & Mean of belief $q_i$ \\[3pt]
Velocity $\dot{\mu}_i$ & Rate of belief change & $\mathbf{M}^{-1}\pi$ \\[3pt]
Momentum $\pi_i$ & Belief velocity $\times$ precision & $\mathbf{M}\dot{\mu}$ \\[3pt]
Mass $M_i$ & Epistemic inertia (precision) & $\bar{\Lambda}_{pi} + \sum_k\beta_{ik}\tilde{\Lambda}_{qk} + \sum_j\beta_{ji}\Lambda_{qi}$ \\[3pt]
Force $f_i$ & Pull toward prior \& consensus & $-\partial F/\partial\mu_i$ \\[3pt]
Kinetic energy $T$ & Cost of rapid belief change & $\frac{1}{2}\pi^T\mathbf{M}^{-1}\pi$ \\[3pt]
Potential energy $V$ & Variational free energy & $F[\mu, \Sigma]$ \\[3pt]
\midrule
\multicolumn{3}{@{}l}{\textit{Covariance (shape) sector:}} \\[3pt]
Position $\Sigma_i$ & Agent $i$'s uncertainty & Covariance of belief $q_i$ \\[3pt]
Momentum $\Pi_i$ & Shape change $\times$ precision$^2$ & $\mathbf{K}\dot{\Sigma}$ \\[3pt]
Shape mass $\mathcal{K}_i$ & Resistance to uncertainty change & $\frac{1}{2}(\Lambda_{qi} \otimes \Lambda_{qi})(1 + \sum\beta)$ \\[3pt]
\midrule
\multicolumn{3}{@{}l}{\textit{Conserved quantities:}} \\[3pt]
Hamiltonian $H$ & Total cognitive energy & $T + V$ \\[3pt]
\bottomrule
\end{tabular}
\end{table}

The identification of mass with precision has several consequences

\begin{enumerate}
    \item \textbf{Confident agents are sluggish}: High precision implies high inertia.  Strongly held beliefs resist change even under social pressure.
    
    \item \textbf{Uncertain agents are responsive}: Low precision means low inertia and therefore agents with diffuse beliefs readily adopt others' views and follows the crowd.
    
    \item \textbf{Social influence is bidirectional}: The recoil term $\sum_j\beta_{ji}\Lambda_i$ shows that influencing others costs inertia.
    
    \item \textbf{Momentum enables overshooting}: Unlike gradient descent, Hamiltonian dynamics can overshoot equilibria and oscillate.
\end{enumerate}

This info-mechanical picture transforms our understanding of socio-cognition from a static optimization problem to a bona-fide dynamical system with conserved quantities, characteristic frequencies, gauge covariance, and rich temporal structure.


%==============================================================================
\subsection{Cognitive Phenomena from Belief Momentum}
\label{sec:cognitive-momentum}
%==============================================================================

The Hamiltonian formulation introduces a quantity absent from standard treatments of Bayesean belief updating: epistemic momentum. Just as physical momentum allows objects to flow past equilibrium, epistemic momentum allows beliefs to overshoot, oscillate, and resist change in ways that pure gradient descent fundamentally cannot capture. 
%------------------------------------------------------------------------------
\subsection{Defining Cognitive Momentum}
%------------------------------------------------------------------------------

\begin{definition}[Cognitive Momentum]

The cognitive momentum of agent $i$ is the product of epistemic mass and belief velocity

\begin{equation}
\boxed{\pi_i = M_i \dot{\mu}_i = \left(\bar{\Lambda}_{pi} + \sum_k \beta_{ik}\tilde{\Lambda}_{qk} + \sum_j \beta_{ji}\Lambda_{qi}\right) \dot{\mu}_i}
\end{equation}

where $\dot{\mu}_i$ is the rate of belief change.

\end{definition}

For an isolated agent with isotropic uncertainty $\Sigma_i = \sigma_i^2 I$, this simplifies to

\begin{equation}
\pi_i = \frac{1}{\sigma_i^2}\dot{\mu}_i = \Lambda_i \dot{\mu}_i
\end{equation}

 Momentum is not simply the velocity of belief. A confident agent (high $\Lambda$) moving slowly has the same momentum as an uncertain agent (low $\Lambda$) moving quickly. This asymmetry has interesting consequences for belief dynamics.

\begin{table}[ht]
\centering
\caption{Components of cognitive momentum and their psychological interpretations.}
\label{tab:momentum-components}
\renewcommand{\arraystretch}{1.3}
\begin{tabular}{@{} l l l @{}}
\toprule
\textbf{Component} & \textbf{Formula} & \textbf{Psychological Role} \\
\midrule
Bare momentum & $\bar{\Lambda}_{pi}\dot{\mu}_i$ & Inertia from prior expectations \\
Social momentum & $\sum_k\beta_{ik}\tilde{\Lambda}_{qk}\dot{\mu}_i$ & Inertia from social embedding \\
Recoil momentum & $\sum_j\beta_{ji}\Lambda_{qi}\dot{\mu}_i$ & Inertia from influencing others \\
\bottomrule
\end{tabular}
\end{table}

%------------------------------------------------------------------------------
\subsection{Confirmation Bias as Momentum}
%------------------------------------------------------------------------------

Presently, research treats confirmation bias as a flaw in evidence evaluation and/or irrationality. Epistemic momentum yields an alternative perspective: confirmation bias is the natural dynamical consequence of beliefs possessing inertia and the underlying informational geometry holding a Fisher metric.

Therefore, we may predict that confident beliefs possess momentum that causes continued motion in their current direction even against mild opposing evidence. The stopping distance for a belief moving at velocity $\dot{\mu}$ against constant opposing force $f$ is then

\begin{equation}
d_{\text{stop}} = \frac{M_i \|\dot{\mu}_i\|^2}{2\|f\|}= \frac{\|\pi_i\|^2}{2M_i\|f\|}
\end{equation}


From energy conservation we have that the initial kinetic energy $\frac{1}{2}\pi^T M^{-1}\pi$ must be dissipated by the work done against force $f$ over distance $d$

\begin{equation}
\frac{1}{2}\pi^T M^{-1}\pi = f \cdot d_{\text{stop}}
\end{equation}
Solving for $d_{\text{stop}}$ gives the result.

This represents a distance in "epistemic" or "informational" space.

As an intuitive example, a person with a strong prior (high $\bar{\Lambda}_p$) who has been moving towards a conclusion/equilibrium (nonzero $\dot{\mu}$) doesn't simply stop when opposing evidence appears. Instead, they continue such that the cognitive momentum carries them beyond where the evidence alone would have lead them. Although this appears as confirmation bias, in our view it is actually belief inertia.

This then leads to a quantitative prediction: the ratio of stopping distances for high-precision ($\Lambda_H$) versus low-precision ($\Lambda_L$) agents is

\begin{equation}
\frac{d_H}{d_L} = \frac{\Lambda_H}{\Lambda_L}
\end{equation}

This implies that a person twice as confident takes twice as long to stop and overshoots twice as far as another.  In principle this can be tested by a clever experimentalist in order to falsify or validate the dynamical framework.

%------------------------------------------------------------------------------
\subsection{Belief Oscillation and Overshooting}
%------------------------------------------------------------------------------

Another prediction of our Hamiltonian epistemic dynamics is oscillation phenomena. Unlike gradient descent, which monotonically approaches equilibrium, Hamiltonian systems can overshoot, oscillate, and decay.

\subsubsection{The Damped Epistemic Oscillator}

By including dissipation (for example, attention deficits, fatigue, etc), the equation of motion becomes

\begin{equation}
M_i\ddot{\mu}_i + \gamma_i\dot{\mu}_i + \nabla_{\mu_i}F = 0
\end{equation}

where $\gamma_i > 0$ is a damping coefficient.  This equation, from the physics perspective, is the well-known driven and damped oscillator.

For small displacements from equilibrium $\mu^*$ we have

\begin{equation}
M_i\ddot{\delta\mu} + \gamma_i\dot{\delta\mu} + K_i\delta\mu = 0
\end{equation}

where $K_i = \nabla^2 F|_{\mu^*}$ represents the belief's "stiffness" (curvature of free energy at equilibrium, completely analogous to a spring).

Once again we arrive at a quantifiable prediction:

In the sub-critical ($\gamma_i < 2\sqrt{K_i M_i}$) regime, beliefs will oscillate around equilibrium with a frequency and decay time given by

\begin{equation}
\boxed{\omega = \sqrt{\frac{K_i}{M_i} - \frac{\gamma_i^2}{4M_i^2}} \approx \sqrt{\frac{\text{Evidence strength}}{\text{Epistemic mass}}}}
\end{equation}

\begin{equation}
\tau = \frac{2M_i}{\gamma_i}
\end{equation}


\subsubsection{Three Dynamical Regimes}

As the standard physics of oscillators show the discriminant $\Delta = \gamma_i^2 - 4K_iM_i$ manifestly determines different behaviors/evolution

\begin{enumerate}
\item \textbf{Over-damped} ($\Delta > 0$): Beliefs decay to equilibrium monotonically without oscillation. This resembles standard Bayesian updating in the literature

\item \textbf{Critically damped} ($\Delta = 0$): This regime exhibits the fastest approach to equilibrium without oscillation. This suggests it may be optimal for rapid learning.

\item \textbf{Under-damped} ($\Delta < 0$): In this regime beliefs oscillate around the equilibrium value, overshooting periodically before equilibrating. This regime produces distinctly non-standard Bayesean dynamics.
\end{enumerate}

As an intuitive and timely example consider an (conspiracy theorist) agent with high precision (strong prior beliefs) and low damping (resistance to evidence). When confronted with strong contradictory evidence the agent will generally exhibit

\begin{enumerate}
\item \textbf{Initial resistance}: High mass $M = \Lambda$ resists the force of evidence
\item \textbf{Acceleration}: Persistent evidence eventually accelerates belief change
\item \textbf{Overshoot}: Momentum carries belief past the truth
\item \textbf{Oscillation}: Belief swings between acceptance and rejection
\item \textbf{Settling}: Damping eventually brings convergence to equilibrium
\end{enumerate}

This pattern (resist, over-correct, oscillate) is consistent with phenomena documented in attitude change and belief correction research \citep{Eagly1993, Lewandowsky2012} but remains unexplained by standard Bayesian models. Here we find a natural and intuitive account.

%------------------------------------------------------------------------------
\subsection{Cognitive Resonance}
%------------------------------------------------------------------------------

Interestingly, general oscillatory systems exhibit resonance phenomena whereby maximum response occurs when the driving frequency matches the system's natural frequency. In our epistemic view this then has direct implications for persuasion and learning.


A prediction presents itself: periodic evidence driving achieves maximum belief change at the agents belief resonance frequency given by

\begin{equation}
\boxed{\omega_{\text{res}} = \sqrt{\frac{K_i}{M_i}} = \sqrt{\frac{\text{Evidence strength} \times \text{Precision}}{\text{Epistemic mass}}}}
\end{equation}


\subsubsection{Amplitude at Resonance}

For example, with sinusoidal forcing $f(t) = f_0\cos(\omega t)$, the steady-state amplitude is shown (in physics/engineering) to be

\begin{equation}
A(\omega) = \frac{f_0/M_i}{\sqrt{(\omega_0^2 - \omega^2)^2 + (\gamma\omega/M_i)^2}}
\end{equation}

where $\omega_0 = \sqrt{K/M}$ is the system's "natural" frequency.

At resonance ($\omega = \omega_{\text{res}} \approx \omega_0$) then, we have

\begin{equation}
A_{\text{max}} = \frac{f_0 M_i}{\gamma_i K_i^{1/2}M_i^{1/2}} = \frac{f_0}{\gamma_i}\sqrt{\frac{M_i}{K_i}}
\end{equation}

Curiously this implies that high-mass (confident) agents have larger resonance amplitudes rather than smaller. While they resist off-resonance forcing, properly timed evidence produces dramatic swings. This prediction then offers myriad applications in psychological/sociological fields (education, advertising, negotiating, therapy, etc).


%------------------------------------------------------------------------------
\subsection{Belief Perseverance}
%------------------------------------------------------------------------------

The characteristic time for a belief to relax toward equilibrium in a social setting is given by

\begin{equation}
\boxed{\tau = \frac{M_i}{\gamma_i} = \frac{\bar{\Lambda}_i + \sum_k\beta_{ik}\tilde{\Lambda}_k + \sum_j\beta_{ji}\Lambda_i}{\gamma_i}}
\end{equation}


High-precision beliefs have long decay times. This suggests phenomena where agents tend to hold onto beliefs even after thorough debunking and evidence to their contrary.

For example, if agent A has precision $\Lambda_A = 10$ and agent B has $\Lambda_B = 1$ (both with equal damping $\gamma$), then

\begin{equation}
\frac{\tau_A}{\tau_B} = \frac{\Lambda_A}{\Lambda_B} = 10
\end{equation}

Agent A's false beliefs persist ten times longer than that of B's, despite identical evidence exposure.

\subsubsection{The Debunking Problem}

Typically debunking assumes beliefs respond instantaneously to evidence yet our theory of epistemic momentum predicts that immediate debunking is ineffective.  The belief should flow past the correction target. Furthermore, repeated debunking, if not properly timed, can lead to amplification (a well studied phenomenon in debunking studies).  A candidate method for debunking, then, is to properly time the belief trajectory before reinforcing the correction.  However, predicting that time scale for a given agent may be difficult.


%------------------------------------------------------------------------------
\subsection{Sociology and Multi-Agent Momentum Transfer}
%------------------------------------------------------------------------------

When agents interact through the attention free energy ($\beta_{ij}$ term), momentum can transfer between beliefs, i.e. one agent's beliefs affects another's. This suggests a system of coupled equations of motion given an attention pattern of a multi-agent system.

\subsubsection{Coupled Equations of Motion}

The full multi-agent dynamics with damping are

\begin{equation}
\boxed{M_i\ddot{\mu}_i + \gamma_i\dot{\mu}_i + \nabla_{\mu_i}F = 0}
\end{equation}

We may expand the gradient as 

\begin{equation}
M_i\ddot{\mu}_i = -\gamma_i\dot{\mu}_i - \bar{\Lambda}_{pi}(\mu_i - \bar{\mu}_i) - \sum_k\beta_{ik}\tilde{\Lambda}_{qk}(\mu_i - \tilde{\mu}_k) - \sum_j\beta_{ji}\Lambda_{qi}\Omega_{ji}^T(\tilde{\mu}_i^{(j)} - \mu_j)
\end{equation}

Then this can be written as

\begin{equation}
\boxed{\underbrace{M_i\ddot{\mu}_i}_{\text{Inertia}} = -\underbrace{\gamma_i\dot{\mu}_i}_{\text{Damping}} - \underbrace{\nabla_{\mu_i}F_{\text{prior}}}_{\text{Prior force}} - \underbrace{\nabla_{\mu_i}F_{\text{consensus}}}_{\text{Social force}}}
\end{equation}

\subsubsection{Momentum Transfer Theorem}

\begin{theorem}[Momentum Transfer Between Agents]

When agent $k$ changes belief, it transfers epistemic momentum to agent $i$ according to

\begin{equation}
\frac{d\pi_i}{dt}\bigg|_{\text{from } k} = -\beta_{ik}\tilde{\Lambda}_{qk}(\mu_i - \tilde{\mu}_k) - \beta_{ki}\Lambda_{qi}\Omega_{ki}^T(\tilde{\mu}_k^{(i)} - \mu_i)
\end{equation}

The total momentum transfer over a given interaction time scale $[0, T]$ is

\begin{equation}
\boxed{\Delta\pi_i = -\int_0^T \left[\beta_{ik}\tilde{\Lambda}_{qk}(\mu_i - \tilde{\mu}_k) + \beta_{ki}\Lambda_{qi}\Omega_{ki}^T(\tilde{\mu}_k^{(i)} - \mu_i)\right] dt}
\end{equation}
\end{theorem}

\subsubsection{Conservation and Non-Conservation}

Without priors and damping, the total momentum is a conserved quantity.

\begin{equation}
\frac{d}{dt}\sum_i \pi_i = 0 \quad \text{(closed system)}
\end{equation}

In contrast,  with priors and damping, momentum is assuredly not conserved. Momentum flows into the environment (the prior) and is then dissipated

\begin{equation}
\frac{d}{dt}\sum_i \pi_i = -\sum_i\gamma_i\dot{\mu}_i - \sum_i\bar{\Lambda}_{pi}(\mu_i - \bar{\mu}_i)
\end{equation}


This allows us to define a momentum current from agent $k$ to agent $i$ as

\begin{equation}
J_{k\rightarrow i} = \beta_{ik}\tilde{\Lambda}_{qk}(\tilde{\mu}_k - \mu_i)
\end{equation}

This satisfies the continuity equation

\begin{equation}
\dot{\pi}_i + \gamma_i\dot{\mu}_i + \bar{\Lambda}_{pi}(\mu_i - \bar{\mu}_i) = \sum_k J_{k\rightarrow i}
\end{equation}

We find that momentum flows from agents with different beliefsvia attention $\beta_{ik}$ and sender precision $\Lambda_{qk}$. High-precision agents are powerful momentum sources as their motion strongly affects coupled neighbors. Howver, their strength is weighted by their relative attentions $\beta_{ij}$

%------------------------------------------------------------------------------
\subsection{Summary}
%------------------------------------------------------------------------------

\begin{table}[ht]
\centering
\caption{Testable predictions from cognitive momentum theory.}
\label{tab:predictions}
\renewcommand{\arraystretch}{1.3}
\begin{tabular}{@{} p{3cm} p{5cm} p{5cm} @{}}
\toprule
\textbf{Phenomenon} & \textbf{Prediction} & \textbf{Experimental Test} \\
\midrule
Confirmation bias & Stopping distance is $\propto$ precision & Measure belief change latency vs. covariance \\[6pt]
Belief oscillation & Under-damped agents overshoot truth and oscillate & Track belief trajectories over time \\[6pt]
Resonance & Optimal persuasion occurs at $\omega_{\text{res}} = \sqrt{K/M}$ & Vary message timing, measure change \\[6pt]
Perseverance & Decay time $\tau = M/\gamma$ & Measure false belief persistence vs. uncertainty \\[6pt]
Social momentum & High-$\Lambda$ agents transfer more momentum & Attention vs. source confidence \\[6pt]
Recoil & Persuaders become harder to persuade & Measure attitude stiffness after persuasion attempts \\
\bottomrule
\end{tabular}
\end{table}

Our epistemic momentum framework unifies seemingly disparate phenomena such as confirmation bias, belief perseverance, oscillation, and social influence into manifestations of a single underlying epistemic Hamiltonian mechanics. Beliefs are not just updated,  they are accelerated. Evidence does not instantly change minds but rather applies an epistemic force. Finally, confident beliefs don't only resist change rather, they possess epistemic inertia that carries them further than evidence alone would have lead them.




\section{Results}
\label{sec:results}

To validate the theoretical predictions of the epistemic momentum framework, we conducted numerical simulations of the epistemic inertia framework. All simulations employed fourth-order Runge-Kutta integration with adaptive step sizes, implemented in Python using SciPy's \texttt{solve\_ivp} routine on an AMD Ryzen 9900x workstation. Energy conservation was continually monitored throughout to ensure numerical stability, with typical energy drift below $10^{-6}$ per unit time in the absence of dissipation. Source code for all simulations is available in our public repository.

\subsection{Damping Regimes in Belief Dynamics}

We examined the three dynamical regimes predicted by the epistemic oscillator model. For fixed precision $\Lambda = 2$ and evidence strength $K = 1$, the critical damping coefficient is $\gamma_c = 2\sqrt{K\Lambda} \approx 2.83$. We simulated belief evolution from an initial displacement $\mu_0 = 1$ with zero initial momentum under over-damped ($\gamma = 3\gamma_c$), critically damped ($\gamma = \gamma_c$), and under-damped ($\gamma = 0.2\gamma_c$) regimes.

\begin{figure}[htbp]
    \centering
    \includegraphics[width=1\linewidth]{damping_regimes.png}
    \caption{\textbf{Three damping regimes in epistemic dynamics.} 
    Numerical integration of the damped belief oscillator (Eq.~\ref{eq:damped_oscillator}) 
    for fixed precision $\Lambda = 2$ and stiffness $K = 1$, with damping coefficients 
    yielding overdamped ($\gamma > 2\sqrt{K\Lambda}$), critically damped 
    ($\gamma = 2\sqrt{K\Lambda}$), and underdamped ($\gamma < 2\sqrt{K\Lambda}$) dynamics. 
    \textbf{Top left:} Belief trajectories showing monotonic decay (overdamped), 
    fastest equilibration (critical), and oscillatory approach with overshooting (underdamped). 
    \textbf{Top right:} Energy dissipation on logarithmic scale. 
    \textbf{Middle row:} Phase portraits ($\mu$ vs.\ $\pi$) revealing the qualitatively 
    distinct attractor geometries. 
    \textbf{Bottom left:} Velocity evolution $\dot{\mu} = \pi/M$. 
    The underdamped regime predicts belief oscillations absent from standard Bayesian updating, 
    representing a novel empirical signature of epistemic momentum.}
    \label{fig:damping_regimes}
\end{figure}


Figure~\ref{fig:damping_regimes} displays the results of these studies. The over-damped regime exhibits typical exponential decay toward equilibrium, qualitatively resembling standard Bayesian updating where beliefs approach the posterior without overshooting. The critically damped case achieves the fastest equilibration reaching the target belief in the least time without exhibiting oscillations.  This represents an optimal learning regime where evidence is incorporated with maximum efficiency. As predicted, the under-damped regime produces damped oscillations as the belief overshoots the equilibrium, reverses direction, and undergoes multiple oscillatory cycles before settling.

The phase space portraits reveal the geometric differences between regimes. Over-damped trajectories spiral inward towards equilibrium, the critically damped trajectories approach the origin along a separatrix, and the under-damped trajectories execute diminishing elliptical orbits characteristic of a damped harmonic oscillator. The under-damped oscillations represent a novel prediction absent from standard Bayesian inference in that beliefs can transiently overshoot the rational posterior and temporarily adopting positions more extreme than warranted by evidence before relaxing to steady equilibrium.

The measured decay times match those of the theoretical predictions. In the under-damped case, we observe a frequency $\omega = \sqrt{K/\Lambda - \gamma^2/4\Lambda^2} \approx 0.69$~rad/time, in exact agreement with theory. The envelope decay time $\tau = 2\Lambda/\gamma \approx 7.1$ matches the observed exponential decay of oscillation amplitude.

\subsection{Momentum Transfer Between Coupled Agents}

To demonstrate that epistemic momentum has genuine mechanical consequences in multi-agent systems, we simulated two coupled agents with precisions $\Lambda_1 = 2$ (the influencer) and $\Lambda_2 = 1$ (the follower), connected through symmetric attention coupling $\beta_{12} = \beta_{21} = 0.5$. Agent~1 was initialized with momentum $\pi_1(0) = 2$ while Agent~2 began at rest.

\begin{figure}[htbp]
    \centering
    \includegraphics[width=1\linewidth]{momentum_transfer.png}
    \caption{\textbf{Momentum transfer and recoil in coupled agents.} 
    Two agents with precisions $\Lambda_1 = 2$ (influencer) and $\Lambda_2 = 1$ (follower) 
    coupled through mutual attention ($\beta_{12} = \beta_{21} = 0.5$). 
    Agent~1 begins with initial momentum $\pi_1(0) = 2$ while Agent~2 starts at rest. 
    \textbf{Top left:} Belief trajectories showing coordinated evolution toward consensus. 
    \textbf{Top center:} Momentum trajectories demonstrating the \emph{recoil effect}---the 
    influencer's momentum decreases as momentum flows to the follower via the consensus 
    coupling. 
    \textbf{Top right:} Total momentum $\pi_1 + \pi_2$ decays due to damping and prior 
    anchoring; in the absence of these dissipative terms, total momentum would be conserved. 
    \textbf{Bottom row:} Individual phase portraits showing the distinct dynamical signatures 
    of influence (Agent~1) versus reception (Agent~2). 
    This simulation demonstrates that social influence has mechanical consequences.  Changing another's mind necessarily affects one's own epistemic trajectory. This is currently under-appreciated.}
    \label{fig:momentum_transfer}
\end{figure}


Figure~\ref{fig:momentum_transfer} presents the resulting coupled dynamics. As Agent~1's momentum drives its belief forward, the consensus coupling exerts an epistemic force on Agent~2, accelerating the follower's belief in the same direction. As an epistemic analog of Newton's third law, Agent~2 simultaneously exerts an equal and opposite force on Agent~1. This produces the recoil effect such that the influencer's momentum decreases as momentum flows to the follower.

The momentum trajectories demonstrate this transfer quantitatively. Agent~1's momentum drops from its initial value of 2.0 to approximately 0.3 at the point of maximum transfer, while Agent~2's momentum rises from zero to a peak of approximately 1.2. The momentum difference $\pi_1 - \pi_2$ decreases monotonically during the interaction phase, confirming directed momentum flow from influencer to follwer.

Total momentum $\pi_1 + \pi_2$ is not conserved due to dissipative damping and prior precision anchoring which act as external forces on the two-agent system. In the limit of vanishing damping and prior strength, total momentum would be conserved.  This is the epistemic analog of an isolated mechanical system. The simulation confirms that social influence is not uni-directional.  Changing another agent's mind necessarily perturbs one's own epistemic trajectory. This has strong implications for understanding persuasion, negotiation, and belief dynamics in social networks.

\subsection{Confirmation Bias as Stopping Distance}

A central prediction of our framework is that confirmation bias emerges naturally from belief inertia. We tested this prediction by simulating pure ballistic motion against constant counter-evidence, measuring the stopping distance as a function of precision.

For various precision values, $\Lambda \in \{0.5, 1.0, 2.0, 4.0, 8.0\}$, we initialized agents with identical velocity $v_0 = 1$ corresponding to momentum $\pi_0 = \Lambda v_0$, and subjected them to constant opposing force $f = 0.5$ thereby representing counter-evidence. The equations of motion reduce to $\Lambda \ddot{\mu} = -f$, yielding a stopping distance $d_{\text{stop}} = \Lambda v_0^2 / 2f$.

\begin{figure}[htbp]
    \centering
    \includegraphics[width=1\linewidth]{stopping_distance.png}
    \caption{\textbf{Confirmation bias as epistemic stopping distance.} 
    Pure ballistic simulation of belief evolution against constant counter-evidence 
    force $f = 0.5$, with initial velocity $v_0 = 1$ and varying precision $\Lambda$. 
    \textbf{Top left:} Belief trajectories showing that higher-precision agents travel 
    further before stopping (circles mark stopping points). 
    \textbf{Top right:} Stopping distance versus precision demonstrates the predicted 
    linear relationship $d_{\text{stop}} = \Lambda v_0^2 / 2f$ , 
    with $R^2 \approx 1$ confirming exact agreement between simulation and theory. 
    \textbf{Bottom left:} Velocity decay $v(t) = v_0 - (f/\Lambda)t$ showing that 
    higher-mass beliefs decelerate more slowly. 
    This result reframes confirmation bias as a natural consequence of epistemic inertia 
    rather than irrational motivated reasoning: an agent twice as confident overshoots 
    exactly twice as far when confronted with contradictory evidence, purely due to 
    the mechanical relationship between mass and momentum.}
    \label{fig:stopping_distance}
\end{figure}


Belief trajectories (Figure~\ref{fig:stopping_distance} ) show that higher-precision agents travel substantially further before stopping. An agent with $\Lambda = 8$ overshoots the starting position by approximately 8 units, while an agent with $\Lambda = 0.5$ travels only 0.5 units representing a 16-fold difference corresponding exactly to the precision ratio we predict.


This result re-frames confirmation bias as a mechanical phenomenon rather than cognitive irrationality. An agent twice as confident (twice the precision) overshoots exactly twice as far when confronted with contradicting evidence not because they irrationally discount evidence, but because their epistemic momentum requires proportionally more opposing force to arrest. Bias, in our view, is simply epistemic inertia and largely unavoidable for epistemic systems.

\subsection{Resonance Under Periodic Evidence}

The oscillator framework further predicts that periodic evidence can drive a resonant amplification of belief oscillations. We simulated this by subjecting an agent ($\Lambda = 2$, $K = 1$, $\gamma = 0.3$) to sinusoidal forcing $f(t) = f_0 \cos(\omega t)$ with amplitude $f_0 = 0.5$ across a range of driving frequencies.

\begin{figure}[htbp]
    \centering
    \includegraphics[width=1\linewidth]{resonance_curve.png}
    \caption{\textbf{Cognitive resonance under periodic evidence.} 
    Steady-state amplitude response of the damped belief oscillator to sinusoidal forcing 
    $f(t) = f_0 \cos(\omega t)$ across driving frequencies $\omega$. 
    Parameters: $\Lambda = 2$, $K = 1$, $\gamma = 0.3$, $f_0 = 0.5$. 
    \textbf{Top:} Resonance curve showing amplitude $A(\omega)$ peaking at the natural 
    frequency $\omega_0 = \sqrt{K/\Lambda}$. 
    Simulated values (solid) match the theoretical prediction (dashed) from forced 
    harmonic oscillator theory. 
    \textbf{Bottom left:} Example steady-state oscillations at frequencies below, at, 
    and above resonance. 
    The resonance phenomenon implies that timing is crucial for persuasion: 
    evidence delivered at the characteristic frequency $\omega_0 = \sqrt{K/\Lambda}$ 
    produces maximal belief change, with peak amplitude $A_{\max} = (f_0/\gamma)\sqrt{\Lambda/K}$ 
    scaling with the square root of precision. 
    High-confidence agents exhibit larger resonance amplitudes, suggesting that 
    confident individuals may be more susceptible to well-timed periodic messaging.}
    \label{fig:resonance}
\end{figure}


Figure~\ref{fig:resonance} shows the steady-state amplitude $A(\omega)$ as a function of the driving frequency. The response exhibits a clear resonance peak at $\omega_{\text{res}} \approx 0.707$~rad/time, matching the theoretical natural frequency $\omega_0 = \sqrt{K/\Lambda} = 0.707$ to within $0.1\%$. The resonance curve shape matches the theoretical amplitude function as a Lorentzian profile with no free parameters, 

\begin{equation}
    A(\omega) = \frac{f_0/\Lambda}{\sqrt{(\omega_0^2 - \omega^2)^2 + (\gamma\omega/\Lambda)^2}}
\end{equation}


The measured peak amplitude $A_{\text{max}} \approx 2.36$ agrees with the theoretical prediction \\ $(f_0/\gamma)\sqrt{\Lambda/K} = 2.36$. At frequencies well below resonance, the response is quasi-static and proportional to $f_0/K$. Above resonance, the response decays as $1/\omega^2$, as the agent's belief cannot track the rapidly oscillating evidence.

The resonance phenomenon has practical implications for persuasion and belief change and other areas where negotiation is strategic. Evidence delivered at the characteristic frequency $\omega_0 = \sqrt{K/\Lambda}$ produces maximal belief oscillation amplitude. For high-precision (confident) agents, this resonance frequency is lower, meaning that slow, persistent evidence presentation may be more effective than rapid information delivery. Conversely, the large resonance amplitude for high-$\Lambda$ agents suggests that confident individuals may be particularly susceptible to well-timed driven, periodic messaging.





\subsection{Belief Perseverance and Decay Time}

Finally, we examined belief perseverance (the persistence of false beliefs after debunking) through precision-dependent decay times. Utilizing a single exponential decay model under constant damping, we measured the characteristic decay time for agents with varying precision.

\begin{figure}[htbp]
    \centering
    \includegraphics[width=1\linewidth]{perseverance_fixed.png}
    \caption{\textbf{Belief perseverance as precision-dependent decay.} 
    First-order belief decay model $\Lambda \, d\mu/dt = -\gamma \mu$ following debunking, 
    with constant damping $\gamma = 1$ and initial false belief $\mu_0 = 2$. 
    \textbf{Top left:} Decay trajectories for varying precision showing exponential 
    relaxation $\mu(t) = \mu_0 \exp(-t/\tau)$ with decay time $\tau = \Lambda/\gamma$. 
    Higher-precision beliefs persist longer despite identical evidence exposure. 
    \textbf{Top right:} Decay time versus precision confirms the linear relationship 
    $\tau \propto \Lambda$, with simulation matching theory exactly. 
    \textbf{Bottom left:} Normalized trajectories $\mu/\mu_0$ versus $t/\tau$ collapse 
    onto the universal curve $e^{-t/\tau}$, demonstrating scale invariance. 
    This result provides a mechanistic account of the ``continued influence effect'': 
    if $\Lambda_A = 10\Lambda_B$, Agent~A's false beliefs persist ten times longer 
    than Agent~B's, explaining why immediate debunking often fails for confident individuals 
    and why misinformation resistance correlates with prior certainty.}
    \label{fig:perseverance}
\end{figure}

Decay trajectories (Figure~\ref{fig:perseverance}) confirm single-exponential relaxation $\mu(t) = \mu_0 \exp(-t/\tau)$ with decay time $\tau = \Lambda/\gamma$ proportional to agent precision. Measured decay times match theoretical predictions exactly. 

The ratio of decay times equals the ratio of precisions such that an agent with $\Lambda = 8$ exhibits a decay time eight times longer than an agent with $\Lambda = 1$. This provides a mechanistic account of the continued influence effect observed in misinformation research \cite{Lewandowsky2012}. Immediate debunking fails not because confident agents are irrational, but again because their epistemic mass requires proportionally longer exposure to corrective evidence.

\subsection{Summary of Quantitative Validation}

Table~\ref{tab:validation} summarizes the agreement between theoretical predictions and simulation results across all five experimental paradigms.

\begin{table}[htbp]
\centering
\caption{Quantitative validation of theoretical predictions.}
\label{tab:validation}
\begin{tabular}{lccl}
\toprule
\textbf{Prediction} & \textbf{Theory} & \textbf{Measured} & \textbf{Agreement} \\
\midrule
Underdamped frequency & $\omega = 0.693$ & $0.693$ & Exact \\
Envelope decay time & $\tau = 7.07$ & $7.08$ & $< 0.2\%$ \\
Stopping distance scaling & $d \propto \Lambda$ & $R^2 = 1.000$ & Exact \\
Resonance frequency & $\omega_0 = 0.707$ & $0.707$ & $< 0.1\%$ \\
Peak resonance amplitude & $A_{\max} = 2.36$ & $2.36$ & Exact \\
Decay time scaling & $\tau \propto \Lambda$ & $R^2 = 1.000$ & Exact \\
\bottomrule
\end{tabular}
\end{table}

The simulations confirm all quantitative predictions of the epistemic momentum framework without parameter adjustment. 


\section{Discussion}

\subsection{Why Was This Overlooked?}

The connection between precision and inertial mass, despite its naturalness, has remained hidden at the intersection of several far-flung disciplines that rarely communicate and interact with meaningful attention patters ($\beta_{ij}$). Psychology has historically focused on static biases and heuristics by cataloging the ways beliefs deviate from normative standards rather than on the temporal dynamics of how and why beliefs change in the manner they do. Typically researchers have been shy to ask "how fast does this belief evolve?" utilizing the proper tools. Neuroscience, meanwhile, focuses on gradient-based descent primarily because neural systems are highly damped.  For example,  synaptic time constants, metabolic constraints, and homeostatic regulation ensure that neural dynamics are overdamped.  This leads researchers to avoid the complexities of oscillatory or momentum-like behavior that might otherwise have been visible. The brain appears to do gradient descent because it operates in a regime where inertial effects are suppressed; not because momentum is absent from the underlying mathematics.  It's been there the whole time.  We may anticipate its utility in other fields which lean on informational systems.

Information geometry, meanwhile, provides the mathematical language for these ideas.  It was developed largely within statistics and machine learning communities, far isolated from psychological or sociological theory. The Fisher metric was studied as an abstract structure on complicated probability spaces, rather than as an inertia tensor governing dynamics. Mostly, the idea that beliefs possess momentum is counter-intuitive at first glance yet obvious in hindsight. We aren't accustomed to thinking of beliefs as probabilistic dynamical variables with velocity and inertia. We "hold" beliefs, we don't "move" them. Yet the mathematics is clear and unambiguous: the second order expansion of the KL divergence simply contains kinetic terms, and ignoring them discards half of reality.

\subsection{Hierarchical Extensions.}

Elsewhere, we have shown that hierarchical emergence of meta-informational systems from networks of interacting agents \citep{Dennis2025b} is a natural consequence of evolution within our theory. When agents collectively achieve sufficient consensus via attention they undergo a renormalization group-like coarse-graining in time, giving rise to emergent meta-agents with their own beliefs, precisions, and gauge frames. These meta-agents then participate in higher-order consensus dynamics, yielding a recursive hierarchy of informational systems. The epistemic inertia and momentum transfer mechanisms developed in the present work provide the microscopic dynamics underlying such emergence: the timescales of belief equilibration ($\tau = M/\gamma$), the conditions for oscillatory versus monotonic consensus formation, and the momentum currents through attention networks all constrain when and how meta-agent coalescence occurs. A complete theory of multi-scale belief dynamics requires integrating both frameworks and should provide an exciting sandbox for sociological and informational systems.  In this theory there is no discrimination between "physical" systems or "abstract" systems - it potentially offers insight to not only human agents but meta-agents of institutions such as economies, governments, societies, and more.


\subsection{Proposed Experimental Tests}

The predictions derived above distinguish the inertial framework from standard first-order Bayesian models. We briefly outline experimental paradigms that could test these predictions however, detailed implementation is left to future work.

\subsubsection{Belief Oscillation Under Strong Counter-Evidence}

\textbf{Prediction:} Underdamped agents should overshoot equilibrium and exhibit non-monotonic belief trajectories when confronted with strong counter-evidence.

This may be studied by measuring participants' prior beliefs and confidence on contentious topics. Then present strong, credible counter-evidence and track belief trajectories via repeated measurements (e.g., Likert scales at 1-minute intervals over 20 minutes). Standard Bayesian models predict monotonic convergence toward the posterior. The inertial framework predicts that high-confidence participants may transiently overshoot, briefly adopting positions more extreme than the evidence warrants before settling to equilibrium.

Any observed non-monotonicity falsifies purely dissipative models.

\subsubsection{Precision-Dependent Relaxation Times}

\textbf{Prediction:} Belief relaxation time $\tau$ scales linearly with prior precision: $\tau = \Lambda/\gamma$ (Eq.~\ref{eq:decay_time}).

Measure prior confidence via incentivized elicitation (e.g., betting procedures, confidence intervals, etc). Following exposure to counter-evidence, measure time to reach stable posterior beliefs. The framework predicts that participants with twice the initial precision require twice the relaxation time, independent of the direction or magnitude of belief change.

Standard models predict relaxation rates depend on evidence strength, rather than prior confidence.

\subsubsection{Resonant Persuasion}

\textbf{Prediction:} Periodic messaging achieves maximum belief change amplitude at the resonance frequency $\omega_{\text{res}} = \sqrt{K/M}$, where $K$ reflects evidence strength and $M = \Lambda$ is epistemic mass.

Deliver persuasive messages at varying intervals (e.g., every 30 seconds, 2 minutes, 5 minutes, 10 minutes) across conditions, holding total exposure constant. Measure final belief change amplitude. Our framework predicts a non-monotonic relationship with a peak at intermediate frequency determined by participant confidence.

Standard models predict monotonic effects of message frequency (more exposure implies more change). Resonance is a signature of a second-order dynamics.

These examples provide empirical methods of discrimination between the inertial framework and other existing first-order models. Observation of oscillation, precision-scaled relaxation, or resonance phenomena would support the identification of epistemic mass with statistical precision; their absence would constrain or falsify the framework.

\subsection{Relation to Existing Models}

Table~\ref{tab:model_comparison} summarizes the qualitative predictions distinguishing the inertial framework from existing approaches to belief dynamics.

\begin{table}[ht]
\centering
\caption{Predictions distinguishing the inertial framework from first-order models (standard Bayesian updating, diffusion models, predictive coding).}
\label{tab:model_comparison}
\begin{tabular}{lcc}
\hline
\textbf{Phenomenon} & \textbf{First-Order Models} & \textbf{This Framework} \\
\hline
Approach to equilibrium & Monotonic & Can oscillate \\
Precision dependence & Weights evidence & Determines inertia \\
Overshooting & Not predicted & Predicted \\
Resonance & Not predicted & Predicted \\
Multi-agent momentum & Absent & Predicted \\
\hline
\end{tabular}
\end{table}

Standard Bayesian updating and its neural implementations \citep{Friston2010} \citep{Ratcliff2008} correspond to first-order, dissipative dynamics. The standard free energy principle emerges as a limiting case of our framework. In the overdamped limit ($\gamma \gg 2\sqrt{KM}$), inertial terms become negligible and dynamics reduce to gradient descent on the free energy landscape as precisely the standard formulation \citep{Friston2010, Bogacz2017}. Novel predictions arise when damping is sufficiently weak such that second-order terms contribute meaningfully to the dynamics.



\subsection{Limitations and Extensions}

Our current theory rests on several simplifying assumptions that future work should relax and pursue. For simplicity and tractability, we've restricted our present attentions to Gaussian beliefs in the quasi-static regime where priors do not evolve meaningfully. While analytically tractable, Gaussians simply cannot capture the multi-modal distributions characteristic of the complexities of human informational systems. The extension to general exponential families is straightforward but the computational complexity increases substantially. We've also assumed weak coupling between agents, treating the consensus terms perturbatively.  It is unclear at present if strong couplings introduce  non-perturbative interaction terms that may qualitatively change the dynamics and potentially enabling a zoo of phase transitions. Additionally, in the present study we have considered belief precision to be quasi-static. While we present simulation results of precision flow we do not consider them analytically here.  Furthermore, as we discuss elsewhere (in other contexts) our full theory treats priors (as well as gauge frames) as dynamical variables, yielding complicated coupled systems on $\mathbb{R}^d \times \mathrm{SPD}(d) \otimes \mathfrak{g}$ where beliefs and priors both evolve.

Several directions present exciting areas of future study. Multi-modal distributions representing conflicting hypotheses would connect to models of cognitive dissonance and attitude ambivalence. Social networks with explicit momentum exchange across varied rich and dynamic attention patters could illuminate phenomena like viral belief propagation, where the velocity of a meme matters as much as its content. Our framework presented here provides mathematical rigor and a fountain of possibilities; detailed experiment, implementation, and re-evaluation of past experimental results represent the next frontiers.




\section{Conclusion}

We have shown that beliefs naturally possess inertia in relation to prior precision. The straightforward identification that epistemic mass equals statistical precision transforms our understanding of belief dynamics provides new tools that extend beyond dissipative gradient flow and into rich Hamiltonian dynamics.

Our theory predicts oscillations, over-shooting, resistance, decay, and resonances in belief dynamics. More fundamentally, it re-frames cognitive biases not as irrationality but instead as unavoidable consequences of belief inertia. Just as physical mass resists acceleration, cognitive precision resists belief change. 

This shift in perspective offers researchers new tools and methods for understanding persuasion, education, therapy, negotiation, and social dynamics. By recognizing that confident beliefs are massive and uncertain beliefs are light, we chart new frontiers in research and socio-psychological understanding.

The mathematics has been hiding in plain sight for decades due to lack of transitive communication between far-flung fields of differential geometry, physics, and informational geometry.  The Fisher information metric has been whispering this entire time that it is actually an inertia tensor for the dynamics of thought.

\subsubsection{Declaration of generative AI and AI-assisted technologies in the manuscript preparation process.}


During the preparation of this work the author used Anthropic Claude Sonnet 4.5 in order to typeset, debug, and program simulation tools. After using this tool/service, the author reviewed and edited the content as needed and takes full responsibility for the content of the published article.



%==============================================================================
% APPENDIX A: HAMILTONIAN MECHANICS ON STATISTICAL MANIFOLDS
%==============================================================================

\appendix
\section{Hamiltonian Mechanics on Statistical Manifolds}
\label{app:hamiltonian}

This appendix derives the complete mass matrix structure for multi-agent belief dynamics, demonstrating that inertial mass emerges as statistical precision. We work in the quasi-static approximation where prior parameters $(\bar{\mu}_i, \bar{\Sigma}_i)$ evolve slowly relative to beliefs $(\mu_i, \Sigma_i)$.

%------------------------------------------------------------------------------
\subsection{Setup and Notation}
%------------------------------------------------------------------------------

Each agent $i$ maintains a belief distribution $q_i = \mathcal{N}(\mu_i, \Sigma_i)$ anchored to a fixed prior $p_i = \mathcal{N}(\bar{\mu}_i, \bar{\Sigma}_i)$. Define:
\begin{align}
\Lambda_{qi} &= \Sigma_i^{-1} & &\text{(belief precision)} \\
\bar{\Lambda}_{pi} &= \bar{\Sigma}_i^{-1} & &\text{(prior precision)} \\
\tilde{\mu}_k &= \Omega_{ik}\mu_k & &\text{(transported mean)} \\
\tilde{\Lambda}_{qk} &= \Omega_{ik}\Lambda_{qk} \Omega_{ik}^T & &\text{(transported precision)}
\end{align}
where $\Omega_{ik} \in \mathrm{SO}(d)$ is the gauge transport operator from agent $k$'s frame to agent $i$'s frame, given by $\Omega_{ik} = e^{\phi_i}e^{-\phi_j}$ with $\phi_i \in \mathfrak{so}(d)$.

The unified free energy functional is:
\begin{equation}
\boxed{F[\{q_i\}] = \sum_i \mathrm{KL}(q_i \| p_i) + \sum_{i,k} \beta_{ik} \, \mathrm{KL}(q_i \| \Omega_{ik}[q_k])}
\end{equation}

%------------------------------------------------------------------------------
\subsection{KL Divergence for Gaussians}
%------------------------------------------------------------------------------

For $q = \mathcal{N}(\mu_q, \Sigma_q)$ and $p = \mathcal{N}(\mu_p, \Sigma_p)$:
\begin{equation}
\mathrm{KL}(q \| p) = \frac{1}{2}\left[ \mathrm{tr}(\Sigma_p^{-1}\Sigma_q) + (\mu_p - \mu_q)^T \Sigma_p^{-1} (\mu_p - \mu_q) - d + \ln\frac{|\Sigma_p|}{|\Sigma_q|} \right]
\end{equation}

%------------------------------------------------------------------------------
\subsection{First Variations (Gradient)}
%------------------------------------------------------------------------------

\subsubsection{Self-Consistency Term: $\mathrm{KL}(q_i \| p_i)$}

\begin{align}
\frac{\partial \mathrm{KL}(q_i \| p_i)}{\partial \mu_i} &= \bar{\Lambda}_{pi}(\mu_i - \bar{\mu}_i) \\[6pt]
\frac{\partial \mathrm{KL}(q_i \| p_i)}{\partial \Sigma_i} &= \frac{1}{2}(\bar{\Lambda}_{pi} - \Lambda_{qi})
\end{align}

\subsubsection{Consensus Term: $\mathrm{KL}(q_i \| \tilde{q}_k)$}

With respect to receiver $i$:
\begin{align}
\frac{\partial \mathrm{KL}(q_i \| \tilde{q}_k)}{\partial \mu_i} &= \tilde{\Lambda}_{qk}(\mu_i - \tilde{\mu}_k) \\[6pt]
\frac{\partial \mathrm{KL}(q_i \| \tilde{q}_k)}{\partial \Sigma_i} &= \frac{1}{2}(\tilde{\Lambda}_{qk} - \Lambda_{qi})
\end{align}

With respect to sender $k$:
\begin{align}
\frac{\partial \mathrm{KL}(q_i \| \tilde{q}_k)}{\partial \mu_k} &= \Lambda_{qk} \Omega_{ik}^T (\tilde{\mu}_k - \mu_i) \\[6pt]
\frac{\partial \mathrm{KL}(q_i \| \tilde{q}_k)}{\partial \Sigma_k} &= \frac{1}{2} \Omega_{ik}^T \left[\tilde{\Lambda}_{qk} - \tilde{\Lambda}_{qk} \Sigma_i \tilde{\Lambda}_{qk} \right] \Omega_{ik}
\end{align}

\subsubsection{Total Gradient}

\begin{equation}
\boxed{\frac{\partial F}{\partial \mu_i} = \bar{\Lambda}_{pi}(\mu_i - \bar{\mu}_i) + \sum_k \beta_{ik} \tilde{\Lambda}_{qk}(\mu_i - \tilde{\mu}_k) + \sum_j \beta_{ji} \Lambda_{qi} \Omega_{ji}^T (\tilde{\mu}_i^{(j)} - \mu_j)}
\end{equation}

where $\tilde{\mu}_i^{(j)} = \Omega_{ji}\mu_i$ is agent $i$'s mean transported into agent $j$'s frame.

%------------------------------------------------------------------------------
\subsection{Second Variations (Hessian = Mass Matrix)}
%------------------------------------------------------------------------------

The Fisher-Rao metric $\mathcal{G} = \partial^2 F / \partial\xi\partial\xi$ serves as the mass matrix.

\subsubsection{Mean Sector: $\partial^2 F / \partial\mu\partial\mu^T$}

\paragraph{Diagonal blocks $(i = k)$:}

From self-consistency:
\begin{equation}
\frac{\partial^2 \mathrm{KL}(q_i \| p_i)}{\partial\mu_i \partial\mu_i^T} = \bar{\Lambda}_{pi}
\end{equation}

From consensus (as receiver):
\begin{equation}
\frac{\partial^2 \mathrm{KL}(q_i \| \tilde{q}_k)}{\partial\mu_i \partial\mu_i^T} = \tilde{\Lambda}_{qk} = \Omega_{ik} \Lambda_{qk} \Omega_{ik}^T
\end{equation}

From consensus (as sender to agent $j$)

\begin{equation}
\frac{\partial^2 \mathrm{KL}(q_j \| \tilde{q}_i)}{\partial\mu_i \partial\mu_i^T} = \Omega_{ji}^T \tilde{\Lambda}_{qi}^{(j)} \Omega_{ji} = \Lambda_{qi}
\end{equation}

\begin{equation}
\boxed{[\mathbf{M}^\mu]_{ii} = \bar{\Lambda}_{pi} + \sum_k \beta_{ik} \tilde{\Lambda}_{qk} + \sum_j \beta_{ji} \Lambda_{qi}}
\end{equation}

\paragraph{Off-diagonal blocks $(i \neq k)$:}

From $\mathrm{KL}(q_i \| \tilde{q}_k)$:
\begin{equation}
\frac{\partial^2 \mathrm{KL}(q_i \| \tilde{q}_k)}{\partial\mu_i \partial\mu_k^T} = -\tilde{\Lambda}_{qk} \Omega_{ik} = -\Omega_{ik} \Lambda_{qk}
\end{equation}

From $\mathrm{KL}(q_k \| \tilde{q}_i)$ (if $k$ also listens to $i$):
\begin{equation}
\frac{\partial^2 \mathrm{KL}(q_k \| \tilde{q}_i)}{\partial\mu_i \partial\mu_k^T} = -\Lambda_{qi} \Omega_{ki}^T
\end{equation}

\begin{equation}
\boxed{[\mathbf{M}^\mu]_{ik} = -\beta_{ik} \Omega_{ik} \Lambda_{qk} - \beta_{ki} \Lambda_{qi} \Omega_{ki}^T \quad (i \neq k)}
\end{equation}

\textbf{Remark:} The mass matrix is symmetric only when $\beta_{ik} = \beta_{ki}$ and $\Omega_{ik} = \Omega_{ki}^T$.

%------------------------------------------------------------------------------
\subsubsection{Covariance Sector: $\partial^2 F / \partial\Sigma\partial\Sigma$}
%------------------------------------------------------------------------------

For matrix-valued variables, we use the directional derivative convention:
\begin{equation}
\frac{\partial^2 f}{\partial\Sigma\partial\Sigma}[V, W] = \lim_{\epsilon \to 0} \frac{1}{\epsilon}\left(\frac{\partial f}{\partial\Sigma}\bigg|_{\Sigma + \epsilon W} - \frac{\partial f}{\partial\Sigma}\bigg|_{\Sigma}\right)[V]
\end{equation}

\paragraph{Key identity:}
\begin{equation}
\frac{\partial}{\partial\Sigma}(\Sigma^{-1}) = -\Sigma^{-1} \otimes \Sigma^{-1}
\end{equation}

\paragraph{Diagonal blocks $(i = k)$:}

From self-consistency:
\begin{equation}
\frac{\partial^2 \mathrm{KL}(q_i \| p_i)}{\partial\Sigma_i \partial\Sigma_i}[V, W] = \frac{1}{2}\mathrm{tr}\left[\Lambda_{qi} V \Lambda_{qi} W\right]
\end{equation}

In tensor notation:
\begin{equation}
\frac{\partial^2 \mathrm{KL}(q_i \| p_i)}{\partial\Sigma_i \partial\Sigma_i} = \frac{1}{2}(\Lambda_{qi} \otimes \Lambda_{qi})
\end{equation}

From consensus (as receiver), the contribution is identical since the derivative is with respect to $\Sigma_i$ appearing in the $-\ln|\Sigma_i|$ term

\begin{equation}
\frac{\partial^2 \mathrm{KL}(q_i \| \tilde{q}_k)}{\partial\Sigma_i \partial\Sigma_i} = \frac{1}{2}(\Lambda_{qi} \otimes \Lambda_{qi})
\end{equation}

From consensus (as sender to agent $j$), we must differentiate the terms involving $\tilde{\Sigma}_i = \Omega_{ji}\Sigma_i\Omega_{ji}^T$:
\begin{equation}
\frac{\partial^2 \mathrm{KL}(q_j \| \tilde{q}_i)}{\partial\Sigma_i \partial\Sigma_i} = \frac{1}{2}(\Omega_{ji}^T \otimes \Omega_{ji}^T)(\tilde{\Lambda}_{qi}^{(j)} \otimes \tilde{\Lambda}_{qi}^{(j)})(\Omega_{ji} \otimes \Omega_{ji}) = \frac{1}{2}(\Lambda_{qi} \otimes \Lambda_{qi})
\end{equation}

where we used $\Omega_{ji}^T \tilde{\Lambda}_{qi}^{(j)} \Omega_{ji} = \Lambda_{qi}$.

\begin{equation}
\boxed{[\mathbf{M}^\Sigma]_{ii} = \frac{1}{2}(\Lambda_{qi} \otimes \Lambda_{qi}) \cdot \left(1 + \sum_k \beta_{ik} + \sum_j \beta_{ji}\right)}
\end{equation}

\paragraph{Off-diagonal blocks $(i \neq k)$:}

From $\mathrm{KL}(q_i \| \tilde{q}_k)$, varying both $\Sigma_i$ and $\Sigma_k$:
\begin{equation}
[\mathbf{M}^\Sigma]_{ik} = -\frac{1}{2}\beta_{ik}(\Omega_{ik}^T \otimes \Omega_{ik}^T)\left(\tilde{\Lambda}_{qk} \Sigma_i \tilde{\Lambda}_{qk} \otimes \tilde{\Lambda}_{qk} + \tilde{\Lambda}_{qk} \otimes \tilde{\Lambda}_{qk} \Sigma_i \tilde{\Lambda}_{qk}\right)(\Omega_{ik} \otimes \Omega_{ik})
\end{equation}

%------------------------------------------------------------------------------
\subsubsection{Mean-Covariance Cross Blocks}
%------------------------------------------------------------------------------

\paragraph{Self-consistency:}
\begin{equation}
\frac{\partial^2 \mathrm{KL}(q_i \| p_i)}{\partial\mu_i \partial\Sigma_i} = 0
\end{equation}

\textbf{Key simplification:} With quasi-static priors, the mean and covariance dynamics decouple at second order for the self-consistency term.

\paragraph{Consensus (cross-agent):}

From $\partial\mathrm{KL}(q_i \| \tilde{q}_k)/\partial\mu_i = \tilde{\Lambda}_{qk}(\mu_i - \tilde{\mu}_k)$, varying $\Sigma_k$:
\begin{equation}
\frac{\partial^2 \mathrm{KL}(q_i \| \tilde{q}_k)}{\partial\mu_i \partial\Sigma_k}[V] = -\Omega_{ik} \Lambda_{qk} V \Lambda_{qk} \Omega_{ik}^T (\mu_i - \tilde{\mu}_k)
\end{equation}

In components:
\begin{equation}
\boxed{\frac{\partial^2 \mathrm{KL}}{\partial(\mu_i)_a \partial(\Sigma_k)_{bc}} = -[\Omega_{ik} \Lambda_{qk}]_{ab} [\Lambda_{qk} \Omega_{ik}^T (\mu_i - \tilde{\mu}_k)]_c}
\end{equation}

This vanishes at consensus ($\mu_i = \tilde{\mu}_k$).

%------------------------------------------------------------------------------
\subsection{Complete Mass Matrix Assembly}
%------------------------------------------------------------------------------

The full state vector is $\xi = (\mu_1, \ldots, \mu_N, \Sigma_1, \ldots, \Sigma_N)$.

\subsubsection{Block Structure}

\begin{equation}
\boxed{
\mathbf{M} = \begin{pmatrix}
\mathbf{M}^\mu & \mathbf{C}^{\mu\Sigma} \\[6pt]
(\mathbf{C}^{\mu\Sigma})^T & \mathbf{M}^\Sigma
\end{pmatrix}
}
\end{equation}

where each block is an $N \times N$ matrix of sub-blocks.

\subsubsection{Explicit Formulae}

\paragraph{Mean sector diagonal:}
\begin{equation}
[\mathbf{M}^\mu]_{ii} = \underbrace{\bar{\Lambda}_{pi}}_{\text{prior anchoring}} + \underbrace{\sum_k \beta_{ik} \Omega_{ik} \Lambda_{qk} \Omega_{ik}^T}_{\text{incoming consensus}} + \underbrace{\sum_j \beta_{ji} \Lambda_{qi}}_{\text{outgoing recoil}}
\end{equation}

\paragraph{Mean sector off-diagonal:}
\begin{equation}
[\mathbf{M}^\mu]_{ik} = -\beta_{ik} \Omega_{ik} \Lambda_{qk} - \beta_{ki} \Lambda_{qi} \Omega_{ki}^T \quad (i \neq k)
\end{equation}

\paragraph{Covariance sector diagonal:}
\begin{equation}
[\mathbf{M}^\Sigma]_{ii} = \frac{1}{2}(\Lambda_{qi} \otimes \Lambda_{qi}) \cdot \left(1 + \sum_k \beta_{ik} + \sum_j \beta_{ji}\right)
\end{equation}

\paragraph{Cross mean-covariance (at consensus):}
\begin{equation}
[\mathbf{C}^{\mu\Sigma}]_{ik} = 0 \quad \text{when } \mu_i = \tilde{\mu}_k
\end{equation}

%------------------------------------------------------------------------------
\subsection{Physical Interpretation}
%------------------------------------------------------------------------------

\subsubsection{Mass as Precision}

The mean-sector mass for agent $i$ is:
\begin{equation}
\boxed{M_i = \bar{\Lambda}_{pi} + \sum_k \beta_{ik} \tilde{\Lambda}_{qk} + \sum_j \beta_{ji} \Lambda_{qi}}
\end{equation}

\begin{itemize}
\item $\bar{\Lambda}_{pi}$: \textbf{Bare mass} --- inertia against deviation from prior
\item $\sum_k \beta_{ik} \tilde{\Lambda}_{qk}$: \textbf{Incoming relational mass} --- inertia from being ``pulled'' by neighbors
\item $\sum_j \beta_{ji} \Lambda_{qi}$: \textbf{Outgoing relational mass} --- inertia from ``pulling'' neighbors (recoil)
\end{itemize}

\subsubsection{Kinetic Energy}

\begin{equation}
\boxed{T = \frac{1}{2}\dot{\mu}^T \mathbf{M}^\mu \dot{\mu} + \frac{1}{2}\mathrm{tr}\left[\mathbf{M}^\Sigma[\dot{\Sigma}, \dot{\Sigma}]\right]}
\end{equation}

The first term gives standard ``particle'' kinetic energy with precision-mass. The second gives ``shape'' kinetic energy on the SPD manifold.

\subsubsection{Inter-Agent Kinetic Coupling}

\begin{equation}
T_{\text{couple}} = \sum_{i < k} \left[-\beta_{ik}\dot{\mu}_i^T \Omega_{ik} \Lambda_{qk} \dot{\mu}_k - \beta_{ki}\dot{\mu}_i^T \Lambda_{qi} \Omega_{ki}^T \dot{\mu}_k\right]
\end{equation}

This represents \textbf{kinetic correlation}: when agent $k$ accelerates, agent $i$ feels a ``drag'' proportional to coupling strength and relative precision.

%------------------------------------------------------------------------------
\subsection{Hamilton's Equations}
%------------------------------------------------------------------------------

With conjugate momenta $\pi = (\pi^\mu, \Pi^\Sigma)$ and Hamiltonian:
\begin{equation}
\boxed{H = \frac{1}{2}\langle\pi, \mathbf{M}^{-1}\pi\rangle + F[\xi]}
\end{equation}

\subsubsection{Equations of Motion}

\begin{align}
\dot{\mu}_i &= \sum_k [\mathbf{M}^{-1}]_{ik}^{\mu\mu} \pi_k^\mu + \sum_k [\mathbf{M}^{-1}]_{ik}^{\mu\Sigma} \Pi_k^\Sigma \\[8pt]
\dot{\Sigma}_i &= \sum_k [\mathbf{M}^{-1}]_{ik}^{\Sigma\mu} \pi_k^\mu + \sum_k [\mathbf{M}^{-1}]_{ik}^{\Sigma\Sigma} \Pi_k^\Sigma
\end{align}

\begin{align}
\dot{\pi}_i^\mu &= -\frac{\partial F}{\partial\mu_i} - \frac{1}{2}\pi^T \frac{\partial \mathbf{M}^{-1}}{\partial\mu_i} \pi \\[8pt]
\dot{\Pi}_i^\Sigma &= -\frac{\partial F}{\partial\Sigma_i} - \frac{1}{2}\pi^T \frac{\partial \mathbf{M}^{-1}}{\partial\Sigma_i} \pi
\end{align}

\subsubsection{Force Terms}

The potential forces are:
\begin{align}
-\frac{\partial F}{\partial\mu_i} &= -\bar{\Lambda}_{pi}(\mu_i - \bar{\mu}_i) - \sum_k \beta_{ik} \tilde{\Lambda}_{qk}(\mu_i - \tilde{\mu}_k) - \sum_j \beta_{ji} \Lambda_{qi} \Omega_{ji}^T(\tilde{\mu}_i^{(j)} - \mu_j) \\[8pt]
-\frac{\partial F}{\partial\Sigma_i} &= -\frac{1}{2}(\bar{\Lambda}_{pi} - \Lambda_{qi}) - \sum_k \frac{\beta_{ik}}{2}(\tilde{\Lambda}_{qk} - \Lambda_{qi}) - \sum_j \frac{\beta_{ji}}{2}\Omega_{ji}^T(\tilde{\Lambda}_{qi}^{(j)} - \tilde{\Lambda}_{qi}^{(j)}\Sigma_j\tilde{\Lambda}_{qi}^{(j)})\Omega_{ji}
\end{align}

The geodesic forces (from metric variation) couple the dynamics across agents.

%------------------------------------------------------------------------------
\subsection{Isotropic Simplification}
%------------------------------------------------------------------------------

With $\Sigma_i = \sigma_{qi}^2 I$, $\bar{\Sigma}_{pi} = \bar{\sigma}_{pi}^2 I$, and $\Omega_{ik} \in \mathrm{SO}(d)$:

\subsubsection{Reduced Variables}
\begin{equation}
\xi_i = (\mu_i, \sigma_i) \in \mathbb{R}^d \times \mathbb{R}^+
\end{equation}

\subsubsection{Scalar Mass}
\begin{equation}
\boxed{m_i = \frac{1}{\bar{\sigma}_{pi}^2} + \sum_k \frac{\beta_{ik}}{\sigma_{qk}^2} + \sum_j \frac{\beta_{ji}}{\sigma_{qi}^2}}
\end{equation}

\textbf{Mass = Total precision} (prior + consensus partners).

\subsubsection{Simplified Hessian}

\begin{align}
[\mathbf{M}^\mu]_{ii} &= \frac{1}{\bar{\sigma}_{pi}^2}I + \sum_k \frac{\beta_{ik}}{\sigma_{qk}^2} \Omega_{ik} \Omega_{ik}^T = \frac{1}{\bar{\sigma}_{pi}^2}I + \sum_k \frac{\beta_{ik}}{\sigma_{qk}^2} I \\[8pt]
[\mathbf{M}^\mu]_{ik} &= -\frac{\beta_{ik}}{\sigma_{qk}^2} \Omega_{ik} - \frac{\beta_{ki}}{\sigma_{qi}^2} \Omega_{ki}^T \quad (i \neq k) \\[8pt]
[\mathbf{M}^\sigma]_{ii} &= \frac{d}{\sigma_{qi}^2}\left(1 + \sum_k \beta_{ik} + \sum_j \beta_{ji}\right) \quad \text{(hyperbolic geometry in log-variance)}
\end{align}

\subsubsection{Force}
\begin{equation}
f_i = -\frac{\mu_i - \bar{\mu}_i}{\bar{\sigma}_{pi}^2} - \sum_k \frac{\beta_{ik}}{\sigma_{qk}^2}(\mu_i - \tilde{\mu}_k) - \sum_j \frac{\beta_{ji}}{\sigma_{qi}^2} \Omega_{ji}^T(\tilde{\mu}_i^{(j)} - \mu_j)
\end{equation}

%------------------------------------------------------------------------------
\subsection{Summary}
%------------------------------------------------------------------------------

\begin{tcolorbox}[title=The Complete Theory]
\textbf{State:} Each agent $i$ has belief $q_i = \mathcal{N}(\mu_i, \Sigma_i)$ with fixed prior $p_i = \mathcal{N}(\bar{\mu}_i, \bar{\Sigma}_i)$.

\textbf{Free Energy:}
\begin{equation}
F = \sum_i \mathrm{KL}(q_i \| p_i) + \sum_{i,k} \beta_{ik}\mathrm{KL}(q_i \| \Omega_{ik}[q_k])
\end{equation}

\textbf{Mass Matrix:}
\begin{equation}
\mathbf{M} = \frac{\partial^2 F}{\partial\xi\partial\xi} = \text{Fisher information} = \text{Precision}
\end{equation}

\textbf{Dynamics:}
\begin{equation}
\dot{\xi} = \mathbf{M}^{-1}\pi, \qquad \dot{\pi} = -\nabla F - \frac{1}{2}\nabla_\xi\langle\pi, \mathbf{M}^{-1}\pi\rangle
\end{equation}

\textbf{Physical Meaning:}
\begin{itemize}
\item Position $\mu_i$ = what agent $i$ believes
\item Momentum $\pi_i$ = rate of belief change $\times$ precision
\item Mass = precision (tight beliefs are heavy)
\item Force = pull toward prior + pull toward consensus
\end{itemize}
\end{tcolorbox}

%==============================================================================
% APPENDIX B: GAUGE FRAME VARIATIONS AND PULLBACK GEOMETRY
%==============================================================================

\section{Gauge Frame Variations and Pullback Geometry}
\label{app:gauge}

The Hamiltonian formulation of belief dynamics reflects deep geometric structure. Each agent's belief space carries a gauge freedom---the choice of coordinate frame in which beliefs are expressed. Physical quantities must be invariant under these gauge transformations, while the dynamics must be covariant. This appendix develops the transformation theory for the mass matrix, momenta, and Hamilton's equations under gauge frame variations.

%------------------------------------------------------------------------------
\subsection{Gauge Structure of Multi-Agent Belief Systems}
%------------------------------------------------------------------------------

\subsubsection{The Principal Bundle}

The geometric setting is a principal $G$-bundle $\pi: P \to \mathcal{C}$ where

\begin{itemize}
    \item $\mathcal{C}$ is the base manifold (agent positions, social network topology)
    \item $G = \mathrm{SO}(d)$ is the gauge group (rotations in belief space)
    \item The fiber $\pi^{-1}(c)$ over each point $c \in \mathcal{C}$ is the space of reference frames
\end{itemize}

Each agent $i$ located at $c_i \in \mathcal{C}$ expresses beliefs in a local frame. The \textbf{transport operator} $\Omega_{ik} \in \mathrm{SO}(d)$ relates agent $k
s frame to agent $i
s frame.

\subsubsection{Gauge Transformations}

A \textbf{gauge transformation} is a smooth assignment of group elements to each agent

\begin{equation}
g: \{1, \ldots, N\} \to \mathrm{SO}(d), \quad i \mapsto g_i
\end{equation}

Under this transformation, belief parameters transform as:
\begin{align}
\mu_i &\mapsto \mu_i' = g_i \mu_i \\
\Sigma_i &\mapsto \Sigma_i' = g_i \Sigma_i g_i^T \\
\Lambda_{qi} &\mapsto \Lambda_{qi}' = g_i \Lambda_{qi} g_i^T
\end{align}

The transport operators transform as:
\begin{equation}
\Omega_{ik} \mapsto \Omega_{ik}' = g_i \Omega_{ik} g_k^{-1}
\end{equation}

This ensures that the transported belief $\tilde{q}_k = \Omega_{ik}[q_k]$ transforms consistently:
\begin{equation}
\tilde{\mu}_k' = g_i \tilde{\mu}_k, \quad \tilde{\Lambda}_{qk}' = g_i \tilde{\Lambda}_{qk} g_i^T
\end{equation}

%------------------------------------------------------------------------------
\subsection{Transformation of the Mass Matrix}
%------------------------------------------------------------------------------

\subsubsection{Mean Sector}

The mean-sector mass matrix transforms as a tensor under gauge transformations.

\paragraph{Diagonal blocks:}
\begin{align}
[\mathbf{M}^\mu]_{ii}' &= \bar{\Lambda}_{pi}' + \sum_k \beta_{ik}\tilde{\Lambda}_{qk}' + \sum_j \beta_{ji}\Lambda_{qi}' \nonumber\\
&= g_i\bar{\Lambda}_{pi} g_i^T + \sum_k \beta_{ik} g_i\tilde{\Lambda}_{qk} g_i^T + \sum_j \beta_{ji} g_i\Lambda_{qi} g_i^T \nonumber\\
&= g_i \left[\bar{\Lambda}_{pi} + \sum_k \beta_{ik}\tilde{\Lambda}_{qk} + \sum_j \beta_{ji}\Lambda_{qi}\right] g_i^T \nonumber\\
&= g_i \, [\mathbf{M}^\mu]_{ii} \, g_i^T
\end{align}

\paragraph{Off-diagonal blocks:}
\begin{align}
[\mathbf{M}^\mu]_{ik}' &= -\beta_{ik}\Omega_{ik}'\Lambda_{qk}' - \beta_{ki}\Lambda_{qi}'(\Omega_{ki}')^T \nonumber\\
&= -\beta_{ik}(g_i\Omega_{ik}g_k^{-1})(g_k\Lambda_{qk} g_k^T) - \beta_{ki}(g_i\Lambda_{qi} g_i^T)(g_k\Omega_{ki}g_i^{-1})^T \nonumber\\
&= -\beta_{ik} g_i\Omega_{ik}\Lambda_{qk} g_k^T - \beta_{ki} g_i\Lambda_{qi}\Omega_{ki}^T g_k^T \nonumber\\
&= g_i \, [\mathbf{M}^\mu]_{ik} \, g_k^T
\end{align}

\paragraph{Block matrix form:}

Define the block-diagonal gauge matrix:
\begin{equation}
\mathbf{G} = \mathrm{diag}(g_1, g_2, \ldots, g_N) \in \mathrm{SO}(d)^N
\end{equation}

Then the full mean-sector mass matrix transforms as:
\begin{equation}
\boxed{(\mathbf{M}^\mu)' = \mathbf{G} \, \mathbf{M}^\mu \, \mathbf{G}^T}
\end{equation}

This is the transformation law for a $(0,2)$-tensor (metric tensor) on the product manifold.

\subsubsection{Covariance Sector}

The covariance-sector mass involves Kronecker products. Under gauge transformation:
\begin{align}
[\mathbf{M}^\Sigma]_{ii}' &= \frac{1}{2}(\Lambda_{qi}' \otimes \Lambda_{qi}') \cdot \left(1 + \sum_k \beta_{ik} + \sum_j \beta_{ji}\right) \nonumber\\
&= \frac{1}{2}(g_i\Lambda_{qi} g_i^T \otimes g_i\Lambda_{qi} g_i^T) \cdot \left(1 + \sum_k \beta_{ik} + \sum_j \beta_{ji}\right) \nonumber\\
&= \frac{1}{2}(g_i \otimes g_i)(\Lambda_{qi} \otimes \Lambda_{qi})(g_i^T \otimes g_i^T) \cdot \left(1 + \sum_k \beta_{ik} + \sum_j \beta_{ji}\right)
\end{align}

The transformation law is:
\begin{equation}
\boxed{(\mathbf{M}^\Sigma)' = (\mathbf{G} \otimes \mathbf{G}) \, \mathbf{M}^\Sigma \, (\mathbf{G}^T \otimes \mathbf{G}^T)}
\end{equation}

\subsubsection{Cross Blocks}

The mean-covariance cross blocks transform as:
\begin{equation}
(\mathbf{C}^{\mu\Sigma})' = \mathbf{G} \, \mathbf{C}^{\mu\Sigma} \, (\mathbf{G}^T \otimes \mathbf{G}^T)
\end{equation}

%------------------------------------------------------------------------------
\subsection{Transformation of Momenta}
%------------------------------------------------------------------------------

For Hamilton's equations to be covariant, momenta must transform contragrediently to positions.

\subsubsection{Mean Momentum}

The mean momentum transforms as a covector:
\begin{equation}
\boxed{(\pi_i^\mu)' = g_i \, \pi_i^\mu}
\end{equation}

This ensures the pairing $\langle\pi^\mu, \dot{\mu}\rangle$ is gauge-invariant:
\begin{equation}
\langle(\pi^\mu)', \dot{\mu}'\rangle = (g_i\pi_i^\mu)^T(g_i\dot{\mu}_i) = (\pi_i^\mu)^T g_i^T g_i \dot{\mu}_i = (\pi_i^\mu)^T\dot{\mu}_i = \langle\pi^\mu, \dot{\mu}\rangle
\end{equation}

\subsubsection{Covariance Momentum}

The covariance momentum $\Pi^\Sigma \in \mathrm{Sym}(d)$ transforms as:
\begin{equation}
\boxed{(\Pi_i^\Sigma)' = g_i \, \Pi_i^\Sigma \, g_i^T}
\end{equation}

The pairing with $\dot{\Sigma}$ uses the trace:
\begin{equation}
\mathrm{tr}[(\Pi^\Sigma)'\dot{\Sigma}'] = \mathrm{tr}[(g_i\Pi_i^\Sigma g_i^T)(g_i\dot{\Sigma}_i g_i^T)] = \mathrm{tr}[\Pi_i^\Sigma\dot{\Sigma}_i]
\end{equation}
where we used cyclicity of the trace and $g_i^T g_i = I$.

%------------------------------------------------------------------------------
\subsection{Covariance of Hamilton's Equations}
%------------------------------------------------------------------------------

\subsubsection{Velocity Equation}

The velocity equation $\dot{\mu} = (\mathbf{M}^\mu)^{-1}\pi^\mu$ transforms as:
\begin{align}
\dot{\mu}' &= ((\mathbf{M}^\mu)')^{-1}(\pi^\mu)' \nonumber\\
&= (\mathbf{G}\mathbf{M}^\mu\mathbf{G}^T)^{-1}\mathbf{G}\pi^\mu \nonumber\\
&= \mathbf{G}^{-T}(\mathbf{M}^\mu)^{-1}\mathbf{G}^{-1}\mathbf{G}\pi^\mu \nonumber\\
&= \mathbf{G}(\mathbf{M}^\mu)^{-1}\pi^\mu \quad \text{(since } \mathbf{G}^{-T} = \mathbf{G} \text{ for SO}(d)\text{)} \nonumber\\
&= \mathbf{G}\dot{\mu}
\end{align}

This confirms $\dot{\mu}$ transforms as a vector: $\dot{\mu}' = \mathbf{G}\dot{\mu}$.

\subsubsection{Force Equation}

The force equation involves the free energy gradient. Under gauge transformation:
\begin{equation}
\left(\frac{\partial F}{\partial\mu_i}\right)' = g_i \frac{\partial F}{\partial\mu_i}
\end{equation}

This follows from the chain rule and the invariance of $F$ under gauge transformations when transport operators transform consistently.

The geodesic force transforms similarly, ensuring full covariance:
\begin{equation}
\boxed{\dot{\pi}' = \mathbf{G}\dot{\pi}}
\end{equation}

%------------------------------------------------------------------------------
\subsection{The Connection and Its Variation}
%------------------------------------------------------------------------------

\subsubsection{Connection 1-Form}

The transport operators $\Omega_{ik}$ encode a discrete connection on the agent network. For agents connected along an edge $e = (i,k)$, define:
\begin{equation}
A_e = \Omega_{ik} \in \mathrm{SO}(d)
\end{equation}

Under gauge transformation:
\begin{equation}
A_e \mapsto A_e' = g_i A_e g_k^{-1}
\end{equation}

This is the discrete analog of the gauge transformation $A \mapsto gAg^{-1} + g\,dg^{-1}$ for continuous connections.

\subsubsection{Curvature}

The curvature around a closed loop $\gamma = (i \to j \to k \to i)$ is:
\begin{equation}
F_\gamma = \Omega_{ij}\Omega_{jk}\Omega_{ki} \in \mathrm{SO}(d)
\end{equation}

This is gauge-covariant: $F_\gamma' = g_i F_\gamma g_i^{-1}$.

A \textbf{flat connection} satisfies $F_\gamma = I$ for all loops, meaning beliefs can be consistently parallel-transported around any cycle. Nonzero curvature represents ``information geometry frustration''---belief frames cannot be consistently aligned around cycles.

\subsubsection{Variation of Connection}

Consider an infinitesimal variation of the connection:
\begin{equation}
\delta\Omega_{ik} = \omega_{ik} \, \Omega_{ik}, \quad \omega_{ik} \in \mathfrak{so}(d)
\end{equation}

The variation of transported precision is:
\begin{equation}
\delta\tilde{\Lambda}_k = \omega_{ik}\tilde{\Lambda}_k + \tilde{\Lambda}_k\omega_{ik}^T = [\omega_{ik}, \tilde{\Lambda}_k]_+
\end{equation}
where $[\cdot,\cdot]_+$ is the anticommutator (since $\omega_{ik}$ is antisymmetric).

%------------------------------------------------------------------------------
\subsection{Variation of the Mass Matrix Under Connection Changes}
%------------------------------------------------------------------------------

\subsubsection{Diagonal Block Variation}

\begin{equation}
\delta[\mathbf{M}^\mu]_{ii} = \sum_k \beta_{ik} \, \delta\tilde{\Lambda}_{qk} = \sum_k \beta_{ik} \, [\omega_{ik}, \tilde{\Lambda}_{qk}]_+
\end{equation}

\subsubsection{Off-Diagonal Block Variation}

\begin{align}
\delta[\mathbf{M}^\mu]_{ik} &= -\beta_{ik} \, \delta(\Omega_{ik}\Lambda_{qk}) - \beta_{ki} \, \delta(\Lambda_{qi}\Omega_{ki}^T) \nonumber\\
&= -\beta_{ik}\omega_{ik}\Omega_{ik}\Lambda_{qk} - \beta_{ki}\Lambda_{qi}(\omega_{ki}\Omega_{ki})^T \nonumber\\
&= -\beta_{ik}\omega_{ik}\Omega_{ik}\Lambda_{qk} + \beta_{ki}\Lambda_{qi}\Omega_{ki}^T\omega_{ki}^T
\end{align}

Using $\omega_{ki}^T = -\omega_{ki}$ (antisymmetry):
\begin{equation}
\boxed{\delta[\mathbf{M}^\mu]_{ik} = -\beta_{ik}\omega_{ik}\Omega_{ik}\Lambda_{qk} - \beta_{ki}\Lambda_{qi}\Omega_{ki}^T\omega_{ki}}
\end{equation}

%------------------------------------------------------------------------------
\subsection{Pullback Geometry}
%------------------------------------------------------------------------------

The \textbf{pullback} of the metric under a map $\phi: \mathcal{Q} \to \mathcal{Q}$ is central to understanding how geometry transforms under coordinate changes or symmetry actions.

\subsubsection{Pullback of the Fisher-Rao Metric}

Let $\phi_g: \mathcal{Q} \to \mathcal{Q}$ be the action of gauge transformation $g$:
\begin{equation}
\phi_g(\mu, \Sigma) = (g\mu, g\Sigma g^T)
\end{equation}

The pullback metric is:
\begin{equation}
(\phi_g^*\mathcal{G})_{(\mu,\Sigma)}(v, w) = \mathcal{G}_{\phi_g(\mu,\Sigma)}(d\phi_g \cdot v, d\phi_g \cdot w)
\end{equation}

For the Fisher-Rao metric, gauge invariance implies:
\begin{equation}
\boxed{\phi_g^*\mathcal{G} = \mathcal{G}}
\end{equation}

The metric is \textbf{gauge-invariant}---this is the geometric content of our transformation laws.

\subsubsection{Horizontal and Vertical Decomposition}

The tangent space at each point decomposes as:
\begin{equation}
T_{(\mu,\Sigma)}\mathcal{Q} = H_{(\mu,\Sigma)} \oplus V_{(\mu,\Sigma)}
\end{equation}

\begin{itemize}
    \item \textbf{Vertical space} $V$: Directions along gauge orbits (pure gauge changes)
    \item \textbf{Horizontal space} $H$: Directions orthogonal to gauge orbits (physical changes)
\end{itemize}

The connection determines the horizontal subspace. A vector $v = (\delta\mu, \delta\Sigma)$ is horizontal if:
\begin{equation}
\mathcal{G}(v, \xi_X) = 0 \quad \forall X \in \mathfrak{so}(d)
\end{equation}
where $\xi_X$ is the vector field generated by $X$.

\subsubsection{Physical (Gauge-Invariant) Quantities}

Only horizontal components of velocities and momenta correspond to physical observables:

\begin{enumerate}
    \item \textbf{Consensus divergence}: $\|\mu_i - \tilde{\mu}_k\|_{\tilde{\Lambda}_{qk}}^2 = (\mu_i - \tilde{\mu}_k)^T\tilde{\Lambda}_{qk}(\mu_i - \tilde{\mu}_k)$
    
    \item \textbf{Free energy}: $F[\{q_i\}]$ is gauge-invariant by construction
    
    \item \textbf{Hamiltonian}: $H = \frac{1}{2}\langle\pi, \mathbf{M}^{-1}\pi\rangle + F$ is gauge-invariant
    
    \item \textbf{Inter-agent KL divergence}: $\mathrm{KL}(q_i \| \Omega_{ik}[q_k])$ is gauge-invariant
\end{enumerate}

%------------------------------------------------------------------------------
\subsection{Gauge-Fixed Dynamics}
%------------------------------------------------------------------------------

For numerical implementation, it is often convenient to work in a fixed gauge.

\subsubsection{Identity Gauge}

Set $g_i = I$ for all agents. Then:
\begin{itemize}
    \item Transport operators $\Omega_{ik}$ are directly the frame transformations
    \item All quantities take their ``bare'' form
    \item Gauge redundancy is eliminated
\end{itemize}

\subsubsection{Consensus-Aligned Gauge}

Choose gauges so that at equilibrium:
\begin{equation}
\Omega_{ik}^* = I \quad \text{(parallel frames at consensus)}
\end{equation}

This simplifies analysis near equilibrium since transported quantities equal untransported ones.

\subsubsection{Principal Axis Gauge}

For each agent, choose $g_i$ to diagonalize $\Sigma_i$:
\begin{equation}
\Sigma_i' = g_i\Sigma_i g_i^T = \mathrm{diag}(\lambda_1^{(i)}, \ldots, \lambda_d^{(i)})
\end{equation}

This separates dynamics along principal axes of uncertainty.

%------------------------------------------------------------------------------
\subsection{Summary: Gauge-Covariant Hamiltonian Mechanics}
%------------------------------------------------------------------------------

\begin{tcolorbox}[title=Gauge Transformation Laws]
\textbf{Positions:}
\begin{align}
\mu_i' &= g_i\mu_i & \Sigma_i' &= g_i\Sigma_i g_i^T
\end{align}

\textbf{Momenta:}
\begin{align}
(\pi_i^\mu)' &= g_i\pi_i^\mu & (\Pi_i^\Sigma)' &= g_i\Pi_i^\Sigma g_i^T
\end{align}

\textbf{Mass Matrix:}
\begin{equation}
\mathbf{M}' = \mathbf{G}\mathbf{M}\mathbf{G}^T
\end{equation}

\textbf{Transport Operators:}
\begin{equation}
\Omega_{ik}' = g_i\Omega_{ik}g_k^{-1}
\end{equation}

\textbf{Hamilton's Equations:} Fully covariant under these transformations.

\textbf{Physical Observables:} Gauge-invariant quantities include $F$, $H$, and all inter-agent divergences.
\end{tcolorbox}



%%%%%%%%%%%%%%%%%%%%%%%%%%%%%%%%%%%%%%%%%%%%%%%%%%%%%%%%%%%%%%%%%%%%%%%
% Appendix: Hamiltonian Mechanics on Statistical Manifolds
% With Explicit Sensory Likelihood
%%%%%%%%%%%%%%%%%%%%%%%%%%%%%%%%%%%%%%%%%%%%%%%%%%%%%%%%%%%%%%%%%%%%%%%

\section{Hamiltonian Mechanics on Statistical Manifolds}
\label{app:hamiltonian}

This appendix derives the complete mass matrix structure for multi-agent belief dynamics with explicit sensory evidence, demonstrating that inertial mass emerges as statistical precision. We work in the quasi-static approximation where prior parameters $(\bar{\mu}_i, \bar{\Sigma}_i)$ evolve slowly relative to beliefs $(\mu_i, \Sigma_i)$.

%-----------------------------------------------------------------------
\subsection{Setup and Notation}
%-----------------------------------------------------------------------

Each agent $i$ maintains a belief distribution $q_i = \mathcal{N}(\mu_i, \Sigma_i)$ anchored to a fixed prior $p_i = \mathcal{N}(\bar{\mu}_i, \bar{\Sigma}_i)$ and receives observations $o_i$ through a likelihood $p(o_i \mid \theta) = \mathcal{N}(o_i; \theta, \Sigma_{o_i})$. Define:
%
\begin{align}
\Lambda_{qi} &= \Sigma_i^{-1} & &\text{(belief precision)} \\
\bar{\Lambda}_{pi} &= \bar{\Sigma}_i^{-1} & &\text{(prior precision)} \\
\Lambda_{o_i} &= \Sigma_{o_i}^{-1} & &\text{(observation precision)} \\
\tilde{\mu}_k &= \Omega_{ik}\mu_k & &\text{(transported mean)} \\
\tilde{\Lambda}_{qk} &= \Omega_{ik}\Lambda_{qk}\Omega_{ik}^T & &\text{(transported precision)}
\end{align}
%
where $\Omega_{ik} \in \mathrm{SO}(d)$ is the gauge transport operator from agent $k$'s frame to agent $i$'s frame, given by $\Omega_{ik} = e^{\phi_i}e^{-\phi_j}$ with $\phi_i \in \mathfrak{so}(d)$.

%-----------------------------------------------------------------------
\subsection{The Extended Free Energy Functional}
%-----------------------------------------------------------------------

The complete variational free energy with explicit sensory evidence is:
%
\begin{equation}
\boxed{
\mathcal{F}[\{q_i\}] = \sum_i D_{\mathrm{KL}}(q_i \| p_i) + \sum_{i,k} \beta_{ik} D_{\mathrm{KL}}(q_i \| \Omega_{ik}[q_k]) - \sum_i \mathbb{E}_{q_i}[\log p(o_i \mid \theta)]
}
\label{eq:full_free_energy}
\end{equation}
%
The three terms represent:
\begin{enumerate}
    \item \textbf{Prior anchoring}: Deviation from internal world-model
    \item \textbf{Social consensus}: Alignment with other agents via gauge-covariant transport
    \item \textbf{Sensory evidence}: Grounding in observations
\end{enumerate}

%-----------------------------------------------------------------------
\subsection{Component Free Energies for Gaussians}
%-----------------------------------------------------------------------

\subsubsection{KL Divergence Between Gaussians}

For $q = \mathcal{N}(\mu_q, \Sigma_q)$ and $p = \mathcal{N}(\mu_p, \Sigma_p)$:
%
\begin{equation}
D_{\mathrm{KL}}(q \| p) = \frac{1}{2}\left[\mathrm{tr}(\Sigma_p^{-1}\Sigma_q) + (\mu_p - \mu_q)^T\Sigma_p^{-1}(\mu_p - \mu_q) - d + \ln\frac{|\Sigma_p|}{|\Sigma_q|}\right]
\label{eq:kl_gaussian}
\end{equation}

\subsubsection{Expected Log-Likelihood}

For the Gaussian likelihood $p(o_i \mid \theta) = \mathcal{N}(o_i; \theta, \Sigma_{o_i})$:
%
\begin{align}
\mathbb{E}_{q_i}[\log p(o_i \mid \theta)] &= -\frac{d}{2}\log(2\pi) - \frac{1}{2}\log|\Sigma_{o_i}| - \frac{1}{2}\mathbb{E}_{q_i}\left[(o_i - \theta)^T\Lambda_{o_i}(o_i - \theta)\right]
\end{align}
%
The quadratic expectation evaluates to:
%
\begin{equation}
\mathbb{E}_{q_i}\left[(o_i - \theta)^T\Lambda_{o_i}(o_i - \theta)\right] = (o_i - \mu_i)^T\Lambda_{o_i}(o_i - \mu_i) + \mathrm{tr}(\Lambda_{o_i}\Sigma_i)
\end{equation}
%
Therefore:
%
\begin{equation}
\boxed{
-\mathbb{E}_{q_i}[\log p(o_i \mid \theta)] = \frac{1}{2}(o_i - \mu_i)^T\Lambda_{o_i}(o_i - \mu_i) + \frac{1}{2}\mathrm{tr}(\Lambda_{o_i}\Sigma_i) + \mathrm{const}
}
\label{eq:neg_log_likelihood}
\end{equation}

%-----------------------------------------------------------------------
\subsection{First Variations (Gradient)}
%-----------------------------------------------------------------------

\subsubsection{Prior Term: $D_{\mathrm{KL}}(q_i \| p_i)$}

\begin{align}
\frac{\partial D_{\mathrm{KL}}(q_i \| p_i)}{\partial \mu_i} &= \bar{\Lambda}_{pi}(\mu_i - \bar{\mu}_i) \label{eq:grad_prior_mu} \\[6pt]
\frac{\partial D_{\mathrm{KL}}(q_i \| p_i)}{\partial \Sigma_i} &= \frac{1}{2}(\bar{\Lambda}_{pi} - \Lambda_{qi}) \label{eq:grad_prior_sigma}
\end{align}

\subsubsection{Consensus Term: $D_{\mathrm{KL}}(q_i \| \tilde{q}_k)$}

With respect to receiver $i$:
\begin{align}
\frac{\partial D_{\mathrm{KL}}(q_i \| \tilde{q}_k)}{\partial \mu_i} &= \tilde{\Lambda}_{qk}(\mu_i - \tilde{\mu}_k) \label{eq:grad_consensus_mu_i} \\[6pt]
\frac{\partial D_{\mathrm{KL}}(q_i \| \tilde{q}_k)}{\partial \Sigma_i} &= \frac{1}{2}(\tilde{\Lambda}_{qk} - \Lambda_{qi}) \label{eq:grad_consensus_sigma_i}
\end{align}

With respect to sender $k$:
\begin{align}
\frac{\partial D_{\mathrm{KL}}(q_i \| \tilde{q}_k)}{\partial \mu_k} &= \Lambda_{qk}\Omega_{ik}^T(\tilde{\mu}_k - \mu_i) \label{eq:grad_consensus_mu_k} \\[6pt]
\frac{\partial D_{\mathrm{KL}}(q_i \| \tilde{q}_k)}{\partial \Sigma_k} &= \frac{1}{2}\Omega_{ik}^T\left[\tilde{\Lambda}_{qk} - \tilde{\Lambda}_{qk}\Sigma_i\tilde{\Lambda}_{qk}\right]\Omega_{ik} \label{eq:grad_consensus_sigma_k}
\end{align}

\subsubsection{Sensory Term: $-\mathbb{E}_{q_i}[\log p(o_i \mid \theta)]$}

\begin{align}
\frac{\partial}{\partial \mu_i}\left[-\mathbb{E}_{q_i}[\log p(o_i \mid \theta)]\right] &= \Lambda_{o_i}(\mu_i - o_i) \label{eq:grad_sensory_mu} \\[6pt]
\frac{\partial}{\partial \Sigma_i}\left[-\mathbb{E}_{q_i}[\log p(o_i \mid \theta)]\right] &= \frac{1}{2}\Lambda_{o_i} \label{eq:grad_sensory_sigma}
\end{align}

\subsubsection{Total Gradient}

\begin{equation}
\boxed{
\frac{\partial \mathcal{F}}{\partial \mu_i} = \bar{\Lambda}_{pi}(\mu_i - \bar{\mu}_i) + \sum_k \beta_{ik}\tilde{\Lambda}_{qk}(\mu_i - \tilde{\mu}_k) + \sum_j \beta_{ji}\Lambda_{qi}\Omega_{ji}^T(\tilde{\mu}_i^{(j)} - \mu_j) + \Lambda_{o_i}(\mu_i - o_i)
}
\label{eq:total_gradient_mu}
\end{equation}
%
where $\tilde{\mu}_i^{(j)} = \Omega_{ji}\mu_i$ is agent $i$'s mean transported into agent $j$'s frame.

\begin{equation}
\frac{\partial \mathcal{F}}{\partial \Sigma_i} = \frac{1}{2}(\bar{\Lambda}_{pi} - \Lambda_{qi}) + \sum_k \frac{\beta_{ik}}{2}(\tilde{\Lambda}_{qk} - \Lambda_{qi}) + \sum_j \frac{\beta_{ji}}{2}\Omega_{ji}^T\left[\tilde{\Lambda}_{qi}^{(j)} - \tilde{\Lambda}_{qi}^{(j)}\Sigma_j\tilde{\Lambda}_{qi}^{(j)}\right]\Omega_{ji} + \frac{1}{2}\Lambda_{o_i}
\label{eq:total_gradient_sigma}
\end{equation}

%-----------------------------------------------------------------------
\subsection{Second Variations (Hessian = Mass Matrix)}
%-----------------------------------------------------------------------

The Fisher-Rao metric $\mathcal{G} = \partial^2\mathcal{F}/\partial\xi\partial\xi$ serves as the mass matrix.

\subsubsection{Mean Sector: $\partial^2\mathcal{F}/\partial\mu\partial\mu^T$}

\paragraph{Diagonal blocks $(i = k)$:}

From prior:
\begin{equation}
\frac{\partial^2 D_{\mathrm{KL}}(q_i \| p_i)}{\partial \mu_i \partial \mu_i^T} = \bar{\Lambda}_{pi}
\end{equation}

From consensus (as receiver):
\begin{equation}
\frac{\partial^2 D_{\mathrm{KL}}(q_i \| \tilde{q}_k)}{\partial \mu_i \partial \mu_i^T} = \tilde{\Lambda}_{qk} = \Omega_{ik}\Lambda_{qk}\Omega_{ik}^T
\end{equation}

From consensus (as sender to agent $j$):
\begin{equation}
\frac{\partial^2 D_{\mathrm{KL}}(q_j \| \tilde{q}_i)}{\partial \mu_i \partial \mu_i^T} = \Omega_{ji}^T\tilde{\Lambda}_{qi}^{(j)}\Omega_{ji} = \Lambda_{qi}
\end{equation}

From sensory evidence:
\begin{equation}
\frac{\partial^2}{\partial \mu_i \partial \mu_i^T}\left[-\mathbb{E}_{q_i}[\log p(o_i \mid \theta)]\right] = \Lambda_{o_i}
\end{equation}

\textbf{Total diagonal mass:}
\begin{equation}
\boxed{
[\mathbf{M}^\mu]_{ii} = \underbrace{\bar{\Lambda}_{pi}}_{\text{prior}} + \underbrace{\sum_k \beta_{ik}\tilde{\Lambda}_{qk}}_{\text{incoming social}} + \underbrace{\sum_j \beta_{ji}\Lambda_{qi}}_{\text{outgoing recoil}} + \underbrace{\Lambda_{o_i}}_{\text{sensory}}
}
\label{eq:mass_diagonal}
\end{equation}

\paragraph{Off-diagonal blocks $(i \neq k)$:}

From $D_{\mathrm{KL}}(q_i \| \tilde{q}_k)$:
\begin{equation}
\frac{\partial^2 D_{\mathrm{KL}}(q_i \| \tilde{q}_k)}{\partial \mu_i \partial \mu_k^T} = -\tilde{\Lambda}_{qk}\Omega_{ik} = -\Omega_{ik}\Lambda_{qk}
\end{equation}

From $D_{\mathrm{KL}}(q_k \| \tilde{q}_i)$ (if $k$ also listens to $i$):
\begin{equation}
\frac{\partial^2 D_{\mathrm{KL}}(q_k \| \tilde{q}_i)}{\partial \mu_i \partial \mu_k^T} = -\Lambda_{qi}\Omega_{ki}^T
\end{equation}

The sensory term does not couple different agents. Therefore:
\begin{equation}
\boxed{
[\mathbf{M}^\mu]_{ik} = -\beta_{ik}\Omega_{ik}\Lambda_{qk} - \beta_{ki}\Lambda_{qi}\Omega_{ki}^T \quad (i \neq k)
}
\label{eq:mass_offdiagonal}
\end{equation}


The mass matrix is symmetric only when $\beta_{ik} = \beta_{ki}$ and $\Omega_{ik} = \Omega_{ki}^T$.


%-----------------------------------------------------------------------
\subsubsection{Covariance Sector: $\partial^2\mathcal{F}/\partial\Sigma\partial\Sigma$}
%-----------------------------------------------------------------------

For matrix-valued variables, we use the directional derivative convention:
\begin{equation}
\frac{\partial^2 f}{\partial \Sigma \partial \Sigma}[V, W] = \lim_{\epsilon \to 0} \frac{1}{\epsilon}\left(\left.\frac{\partial f}{\partial \Sigma}\right|_{\Sigma + \epsilon W} - \left.\frac{\partial f}{\partial \Sigma}\right|_{\Sigma}\right)[V]
\end{equation}

\paragraph{Key identity:}
\begin{equation}
\frac{\partial}{\partial \Sigma}(\Sigma^{-1}) = -\Sigma^{-1} \otimes \Sigma^{-1}
\end{equation}

\paragraph{Diagonal blocks $(i = k)$:}

From prior:
\begin{equation}
\frac{\partial^2 D_{\mathrm{KL}}(q_i \| p_i)}{\partial \Sigma_i \partial \Sigma_i}[V, W] = \frac{1}{2}\mathrm{tr}\left[\Lambda_{qi}V\Lambda_{qi}W\right]
\end{equation}

In tensor notation:
\begin{equation}
\frac{\partial^2 D_{\mathrm{KL}}(q_i \| p_i)}{\partial \Sigma_i \partial \Sigma_i} = \frac{1}{2}(\Lambda_{qi} \otimes \Lambda_{qi})
\end{equation}

From consensus (as receiver and sender), identical contributions arise.

\textbf{Critical observation:} The sensory term $\frac{1}{2}\mathrm{tr}(\Lambda_{o_i}\Sigma_i)$ is \textit{linear} in $\Sigma_i$, so its second derivative \textbf{vanishes}:
\begin{equation}
\frac{\partial^2}{\partial \Sigma_i \partial \Sigma_i}\left[\mathrm{tr}(\Lambda_{o_i}\Sigma_i)\right] = 0
\end{equation}

Therefore:
\begin{equation}
\boxed{
[\mathbf{M}^\Sigma]_{ii} = \frac{1}{2}(\Lambda_{qi} \otimes \Lambda_{qi}) \cdot \left(1 + \sum_k \beta_{ik} + \sum_j \beta_{ji}\right)
}
\label{eq:mass_covariance}
\end{equation}

The sensory precision $\Lambda_{o_i}$ does \textbf{not} contribute to the covariance-sector mass.

%-----------------------------------------------------------------------
\subsubsection{Mean-Covariance Cross Blocks}
%-----------------------------------------------------------------------

\paragraph{Prior term:}
\begin{equation}
\frac{\partial^2 D_{\mathrm{KL}}(q_i \| p_i)}{\partial \mu_i \partial \Sigma_i} = 0
\end{equation}

\paragraph{Sensory term:} The sensory free energy decomposes as:
\begin{itemize}
    \item Quadratic in $\mu_i$: $(o_i - \mu_i)^T\Lambda_{o_i}(o_i - \mu_i)$
    \item Linear in $\Sigma_i$: $\mathrm{tr}(\Lambda_{o_i}\Sigma_i)$
\end{itemize}
These are independent, so:
\begin{equation}
[\mathbf{C}^{\mu\Sigma}]_{ii}^{\text{sensory}} = 0
\end{equation}

\paragraph{Consensus (cross-agent):} From $\partial D_{\mathrm{KL}}(q_i \| \tilde{q}_k)/\partial \mu_i = \tilde{\Lambda}_{qk}(\mu_i - \tilde{\mu}_k)$, varying $\Sigma_k$:
\begin{equation}
\frac{\partial^2 D_{\mathrm{KL}}(q_i \| \tilde{q}_k)}{\partial \mu_i \partial \Sigma_k}[V] = -\Omega_{ik}\Lambda_{qk}V\Lambda_{qk}\Omega_{ik}^T(\mu_i - \tilde{\mu}_k)
\end{equation}

This vanishes at consensus ($\mu_i = \tilde{\mu}_k$):
\begin{equation}
[\mathbf{C}^{\mu\Sigma}]_{ik} = 0 \quad \text{when } \mu_i = \tilde{\mu}_k
\end{equation}

%-----------------------------------------------------------------------
\subsection{Complete Mass Matrix Assembly}
%-----------------------------------------------------------------------

The full state vector is $\xi = (\mu_1, \ldots, \mu_N, \Sigma_1, \ldots, \Sigma_N)$.

\subsubsection{Block Structure}

\begin{equation}
\mathbf{M} = \begin{pmatrix} \mathbf{M}^\mu & \mathbf{C}^{\mu\Sigma} \\ (\mathbf{C}^{\mu\Sigma})^T & \mathbf{M}^\Sigma \end{pmatrix}
\end{equation}
where each block is an $N \times N$ matrix of sub-blocks.

\subsubsection{Explicit Formulae}

\paragraph{Mean sector diagonal:}
\begin{equation}
[\mathbf{M}^\mu]_{ii} = \underbrace{\bar{\Lambda}_{pi}}_{\text{prior anchoring}} + \underbrace{\sum_k \beta_{ik}\Omega_{ik}\Lambda_{qk}\Omega_{ik}^T}_{\text{incoming consensus}} + \underbrace{\sum_j \beta_{ji}\Lambda_{qi}}_{\text{outgoing recoil}} + \underbrace{\Lambda_{o_i}}_{\text{sensory grounding}}
\label{eq:mass_full_diagonal}
\end{equation}

\paragraph{Mean sector off-diagonal:}
\begin{equation}
[\mathbf{M}^\mu]_{ik} = -\beta_{ik}\Omega_{ik}\Lambda_{qk} - \beta_{ki}\Lambda_{qi}\Omega_{ki}^T \quad (i \neq k)
\end{equation}

\paragraph{Covariance sector diagonal:}
\begin{equation}
[\mathbf{M}^\Sigma]_{ii} = \frac{1}{2}(\Lambda_{qi} \otimes \Lambda_{qi}) \cdot \left(1 + \sum_k \beta_{ik} + \sum_j \beta_{ji}\right)
\end{equation}

\paragraph{Cross mean-covariance (at consensus):}
\begin{equation}
[\mathbf{C}^{\mu\Sigma}]_{ik} = 0 \quad \text{when } \mu_i = \tilde{\mu}_k
\end{equation}

%-----------------------------------------------------------------------
\subsection{Physical Interpretation}
%-----------------------------------------------------------------------

\subsubsection{Mass as Precision}

The mean-sector effective mass for agent $i$ is:
\begin{equation}
\boxed{
M_i = \bar{\Lambda}_{pi} + \sum_k \beta_{ik}\tilde{\Lambda}_{qk} + \sum_j \beta_{ji}\Lambda_{qi} + \Lambda_{o_i}
}
\label{eq:effective_mass}
\end{equation}

\begin{itemize}
    \item $\bar{\Lambda}_{pi}$: \textbf{Bare mass} --- inertia against deviation from prior
    \item $\sum_k \beta_{ik}\tilde{\Lambda}_{qk}$: \textbf{Incoming relational mass} --- inertia from being ``pulled'' by neighbors
    \item $\sum_j \beta_{ji}\Lambda_{qi}$: \textbf{Outgoing relational mass} --- inertia from ``pulling'' neighbors (recoil)
    \item $\Lambda_{o_i}$: \textbf{Sensory mass} --- inertia from grounding in observations
\end{itemize}

\subsubsection{Asymmetry of Sensory Contribution}

The sensory precision $\Lambda_{o_i}$ contributes to:
\begin{enumerate}
    \item The mean-sector mass (Eq.~\ref{eq:mass_full_diagonal})
    \item The mean-sector force (Eq.~\ref{eq:total_gradient_mu})
\end{enumerate}

But \textbf{not} to:
\begin{enumerate}
    \item The covariance-sector mass (Eq.~\ref{eq:mass_covariance})
\end{enumerate}

This asymmetry arises because the sensory term is quadratic in $\mu$ but only linear in $\Sigma$.

\subsubsection{Kinetic Energy}

\begin{equation}
T = \frac{1}{2}\dot{\mu}^T\mathbf{M}^\mu\dot{\mu} + \frac{1}{2}\mathrm{tr}\left[\mathbf{M}^\Sigma[\dot{\Sigma}, \dot{\Sigma}]\right]
\end{equation}

The first term gives standard ``particle'' kinetic energy with precision-mass. The second gives ``shape'' kinetic energy on the SPD manifold.

%-----------------------------------------------------------------------
\subsection{The Hamiltonian}
%-----------------------------------------------------------------------

With conjugate momenta $\pi = (\pi^\mu, \Pi^\Sigma)$ and Hamiltonian:
\begin{equation}
\boxed{
H = \frac{1}{2}\langle\pi, \mathbf{M}^{-1}\pi\rangle + \mathcal{F}[\xi]
}
\label{eq:hamiltonian}
\end{equation}

%-----------------------------------------------------------------------
\subsection{Hamilton's Equations}
%-----------------------------------------------------------------------

\subsubsection{Equations of Motion}

\begin{align}
\dot{\mu}_i &= \sum_k [\mathbf{M}^{-1}]_{ik}^{\mu\mu}\pi_k^\mu + \sum_k [\mathbf{M}^{-1}]_{ik}^{\mu\Sigma}\Pi_k^\Sigma \label{eq:hamilton_mu} \\[6pt]
\dot{\Sigma}_i &= \sum_k [\mathbf{M}^{-1}]_{ik}^{\Sigma\mu}\pi_k^\mu + \sum_k [\mathbf{M}^{-1}]_{ik}^{\Sigma\Sigma}\Pi_k^\Sigma \label{eq:hamilton_sigma} \\[6pt]
\dot{\pi}_i^\mu &= -\frac{\partial \mathcal{F}}{\partial \mu_i} - \frac{1}{2}\pi^T\frac{\partial \mathbf{M}^{-1}}{\partial \mu_i}\pi \label{eq:hamilton_pi_mu} \\[6pt]
\dot{\Pi}_i^\Sigma &= -\frac{\partial \mathcal{F}}{\partial \Sigma_i} - \frac{1}{2}\pi^T\frac{\partial \mathbf{M}^{-1}}{\partial \Sigma_i}\pi \label{eq:hamilton_pi_sigma}
\end{align}

\subsubsection{Force Decomposition}

The potential forces decompose into four physically distinct contributions:
\begin{equation}
\boxed{
-\frac{\partial \mathcal{F}}{\partial \mu_i} = \underbrace{-\bar{\Lambda}_{pi}(\mu_i - \bar{\mu}_i)}_{\text{prior restoring}} \underbrace{- \sum_k \beta_{ik}\tilde{\Lambda}_{qk}(\mu_i - \tilde{\mu}_k)}_{\text{consensus}} \underbrace{- \sum_j \beta_{ji}\Lambda_{qi}\Omega_{ji}^T(\tilde{\mu}_i^{(j)} - \mu_j)}_{\text{reciprocal}} \underbrace{- \Lambda_{o_i}(\mu_i - o_i)}_{\text{sensory evidence}}
}
\label{eq:force_decomposition}
\end{equation}

The geodesic force $f_i^{\text{geo}} = -\frac{1}{2}\sum_{jkl}(\pi_j^\mu)^T\frac{\partial[\mathbf{M}^{-1}]_{jk}^{\mu\mu}}{\partial \mu_i}\pi_k^\mu$ encodes manifold curvature.

\subsubsection{Compact Form}

\begin{equation}
\boxed{
\begin{aligned}
\dot{\xi} &= \mathbf{M}^{-1}\pi \\
\dot{\pi} &= -\nabla\mathcal{F} - \frac{1}{2}\nabla_\xi\langle\pi, \mathbf{M}^{-1}\pi\rangle
\end{aligned}
}
\end{equation}

with $dH/dt = 0$ along trajectories.

%-----------------------------------------------------------------------
\subsection{Damped Dynamics}
%-----------------------------------------------------------------------

Including dissipation yields:
\begin{equation}
M_i\ddot{\mu}_i + \gamma_i\dot{\mu}_i + \nabla_{\mu_i}\mathcal{F} = 0
\label{eq:damped_dynamics}
\end{equation}

For small displacements from equilibrium with stiffness $K_i = \nabla^2\mathcal{F}|_{\mu^*}$:
\begin{equation}
M_i\ddot{\delta\mu} + \gamma_i\dot{\delta\mu} + K_i\delta\mu = 0
\end{equation}

The discriminant $\Delta = \gamma_i^2 - 4K_iM_i$ determines three regimes:
\begin{itemize}
    \item \textbf{Overdamped} ($\Delta > 0$): Monotonic decay, standard Bayesian updating
    \item \textbf{Critically damped} ($\Delta = 0$): Fastest equilibration
    \item \textbf{Underdamped} ($\Delta < 0$): Oscillatory approach with overshooting
\end{itemize}

%-----------------------------------------------------------------------
\subsection{Momentum Current with Sensory Coupling}
%-----------------------------------------------------------------------

Between agents, the momentum current is:
\begin{equation}
J_{k \to i} = \beta_{ik}\tilde{\Lambda}_{qk}(\tilde{\mu}_k - \mu_i)
\end{equation}

The continuity equation becomes:
\begin{equation}
\dot{\pi}_i + \gamma_i\dot{\mu}_i + \bar{\Lambda}_{pi}(\mu_i - \bar{\mu}_i) + \Lambda_{o_i}(\mu_i - o_i) = \sum_k J_{k \to i}
\end{equation}

The sensory term $\Lambda_{o_i}(\mu_i - o_i)$ acts as an additional ``anchoring force'' that grounds the agent in observations, distinct from the social momentum currents.

%-----------------------------------------------------------------------
\subsection{Summary}
%-----------------------------------------------------------------------

\begin{tcolorbox}[colback=gray!5,colframe=gray!75,title=The Complete Theory with Sensory Evidence]

\textbf{State:} Each agent $i$ has belief $q_i = \mathcal{N}(\mu_i, \Sigma_i)$ with fixed prior $p_i = \mathcal{N}(\bar{\mu}_i, \bar{\Sigma}_i)$ and observations $o_i$ with precision $\Lambda_{o_i}$.

\textbf{Free Energy:}
\begin{equation}
\mathcal{F} = \sum_i D_{\mathrm{KL}}(q_i \| p_i) + \sum_{i,k} \beta_{ik} D_{\mathrm{KL}}(q_i \| \Omega_{ik}[q_k]) - \sum_i \mathbb{E}_{q_i}[\log p(o_i \mid \theta)]
\end{equation}

\textbf{Mass Matrix:}
\begin{equation}
\mathbf{M} = \frac{\partial^2 \mathcal{F}}{\partial \xi \partial \xi} = \text{Fisher information} = \text{Precision}
\end{equation}

\textbf{Effective Mass:}
\begin{equation}
M_i = \bar{\Lambda}_{pi} + \sum_k \beta_{ik}\tilde{\Lambda}_{qk} + \sum_j \beta_{ji}\Lambda_{qi} + \Lambda_{o_i}
\end{equation}

\textbf{Dynamics:}
\begin{equation}
\dot{\xi} = \mathbf{M}^{-1}\pi, \qquad \dot{\pi} = -\nabla\mathcal{F} - \frac{1}{2}\nabla_\xi\langle\pi, \mathbf{M}^{-1}\pi\rangle
\end{equation}

\textbf{Physical Meaning:}
\begin{itemize}
    \item Position $\mu_i$ = what agent $i$ believes
    \item Momentum $\pi_i$ = rate of belief change $\times$ precision
    \item Mass = precision (tight beliefs are heavy)
    \item Force = pull toward prior + consensus + observations
\end{itemize}

\end{tcolorbox}


















\end{document}



\bibliographystyle{apalike}
\bibliography{references}

\end{document}