\documentclass[11pt]{letter}
\usepackage[margin=1in]{geometry}
\usepackage{hyperref}

\signature{Robert C. Dennis, Ph.D.\\Independent Researcher\\Leander, Texas\\cdenn016@gmail.com}

\address{Robert C. Dennis\\Leander, TX 78641\\cdenn016@gmail.com}

\begin{document}

\begin{letter}{Editorial Office\\Perspectives on Psychological Science}

\opening{Dear Editors,}

I am pleased to submit my manuscript, ``The Inertia of Belief,'' for consideration as an article in \textit{Perspectives on Psychological Science}.

This manuscript addresses a longstanding puzzle: why do phenomenological mass/spring models of belief dynamics work so well empirically despite lacking theoretical justification? Researchers from Kaplowitz and Fink's attitude oscillation studies to modern opinion dynamics have successfully modeled beliefs as objects with inertial mass and momentum, yet no one has explained \textit{why} beliefs should behave this way.

I show that belief inertia emerges naturally from a second-order expansion of the variational free energy functional on statistical manifolds in a gauge-theoretic setting. The Fisher information metric, which measures statistical distinguishability, simultaneously provides an inertial mass tensor for belief dynamics. This single insight unifies several documented but theoretically orphaned phenomena:

\begin{itemize}
    \item Attitude oscillation and overshoot in persuasion research
    \item Confirmation bias as momentum rather than irrationality
    \item Belief perseverance scaling with prior confidence
    \item Resonance effects in periodic messaging
\end{itemize}

Crucially, the framework extends to multi-agent systems, providing an entirely novel theoretical account of \textit{social} momentum. I show that influence has mechanical consequences: changing another's mind necessarily perturbs one's own epistemic trajectory. Momentum transfers between coupled agents, leaders accumulate inertial mass from their followers' attention, and influence hierarchies naturally produce epistemic stratification. These predictions—derived from first principles rather than fitted post hoc—offer new explanatory purchase on phenomena from groupthink to the rigidity of public figures.

The framework generates falsifiable predictions distinguishing it from standard Bayesian models, including precision-scaled relaxation times and characteristic resonance frequencies. I validate these predictions computationally and show that existing data (e.g., the helicopter task) fall in parameter regimes where the theory reduces to familiar gradient descent—explaining why first-order models succeed where they do.

I believe this work is well-suited for \textit{Perspectives} because it offers a genuinely new theoretical lens on belief dynamics with broad implications across cognitive, social, and clinical psychology. The central claim---that confident beliefs are literally heavier---reframes cognitive biases as geometric necessities rather than failures of rationality.

This manuscript has not been published elsewhere and is not under consideration at another journal. Earlier versions received desk rejections from \textit{Entropy} and \textit{Journal of Mathematical Psychology} (no reviews available); I interpret this as the work falling between disciplinary boundaries, which \textit{Perspectives}' interdisciplinary scope may better accommodate.

Simulation code and data are available at: \url{https://github.com/cdenn016/Hamiltonian-VFE}

Thank you for considering this submission.

\closing{Sincerely,}

\end{letter}
\end{document}
